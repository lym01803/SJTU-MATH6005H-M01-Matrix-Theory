\documentclass{article}
\usepackage{ctex}
\usepackage{amsmath,amscd,amsbsy,amssymb,latexsym,url,bm,amsthm}
\usepackage{epsfig,graphicx,subfigure}
\usepackage{enumitem,balance,mathtools}
\usepackage{wrapfig}
\usepackage{mathrsfs, euscript}
\usepackage[usenames]{xcolor}
\usepackage{hyperref}
\usepackage{caption}
\usepackage{setspace}
%\usepackage{subcaption}
\usepackage{float}
\usepackage{listings}
%\usepackage{enumerate}
%\usepackage{algorithm}
%\usepackage{algorithmic}
%\usepackage[vlined,ruled,commentsnumbered,linesnumbered]{algorithm2e}
\usepackage[ruled,lined,boxed,linesnumbered]{algorithm2e}
\usepackage{tikz}

\newtheorem{theorem}{Theorem}[section]
\newtheorem{lemma}[theorem]{Lemma}
\newtheorem{proposition}[theorem]{Proposition}
\newtheorem{corollary}[theorem]{Corollary}
\newtheorem{exercise}{Exercise}[section]
\newtheorem*{solution}{Solution}

\renewcommand{\thefootnote}{\fnsymbol{footnote}}
\renewenvironment{solution}[1][Solution]{~\\ \textbf{#1.}}{~\\}

\newcommand{\prob}{\mathtt{Pr}}

\newcommand{\postscript}[2]
    {\setlength{\epsfxsize}{#2\hsize}
    \centerline{\epsfbox{#1}}}

\renewcommand{\baselinestretch}{1.0}
\SetKwFor{Function}{function}{:}{end}
\setlength{\oddsidemargin}{-0.365in}
\setlength{\evensidemargin}{-0.365in}
\setlength{\topmargin}{-0.3in}
\setlength{\headheight}{0in}
\setlength{\headsep}{0in}
\setlength{\textheight}{10.1in}
\setlength{\textwidth}{7in}

\title{矩阵理论 作业3}
\author{刘彦铭\ \ 学号:122033910081}
\date{编辑日期:\ \today}

\begin{document}

\maketitle

2.1节:3;

2.2节:1,2,3

\begin{spacing}{1.5}
\begin{itemize}

    \item 2.1 - 习题3
    
    对矩阵 $A$ 模拟高斯消元过程可以知道,需要将原第2行置于第1行,将原第1行置于第2行,于是可以得到可行的置换矩阵 $P = \left(\begin{array}{ccc}0&1&0\\1&0&0\\0&0&1\end{array}\right)$
    
    对 $PA$ 模拟高斯消元过程可得 :
    $$
    \begin{array}{ll}
        PA = \left(\begin{array}{ccc}1&2&3\\0&1&1\\1&2&4\end{array}\right)
           &= \left(\begin{array}{ccc}1&0&0\\0&1&0\\-1&0&1\end{array}\right)^{-1}\times 
              \left(\begin{array}{ccc}1&2&3\\0&1&1\\0&0&1\end{array}\right)\\
           &= \left(\begin{array}{ccc}1&0&0\\0&1&0\\1&0&1\end{array}\right)\times 
              \left(\begin{array}{ccc}1&2&3\\0&1&1\\0&0&1\end{array}\right)\\
    \end{array}
    $$

    于是 $L = \left(\begin{array}{ccc}1&0&0\\0&1&0\\1&0&1\end{array}\right)$, $U=\left(\begin{array}{ccc}1&2&3\\0&1&1\\0&0&1\end{array}\right)$.

    \item 2.2 - 习题1
    
    交换1,3两列,并选取前两列作为列向量的极大无关组。有:$P = \left(\begin{array}{cccc}0&0&1&0\\0&1&0&0\\1&0&0&0\\0&0&0&1\end{array}\right)$
    $AP = \left(\begin{array}{cccc}1&0&2&1\\-1&-1&-1&-1\\0&-1&1&0\\-1&-2&0&-1\end{array}\right)$
    对前两列做Schmidt正交化,再单位化,得到:

    $$
    AP = \left(\begin{array}{cc}
        \dfrac{1}{\sqrt 3} & \dfrac{1}{\sqrt 3}\\
        -\dfrac{1}{\sqrt 3} & 0\\
        0 & \dfrac{1}{\sqrt 3}\\
        -\dfrac{1}{\sqrt 3} & \dfrac{1}{\sqrt 3}\\
    \end{array}\right)
    \times 
    \left(\begin{array}{cccc}
        \sqrt{3} & \sqrt{3} & \sqrt{3} & \sqrt{3}\\
        0 & -\sqrt{3} & \sqrt{3} & 0\\
    \end{array}\right)
    = QR
    $$

    \item 2.2 - 习题2
    
    $A = \left(\begin{array}{cccc}2&0&1&1\\-1&-1&-1&-1\\1&-1&0&0\\0&-2&-1&-1\end{array}\right)$.
    对于 $A$ 的第一列, $A_{:,1} = (-2, -1, 1, 0)^\top$, 我们希望构造 酉矩阵 $U_1$ 使得 $U_1 A_{:,1} = (||A_{:,1}||, 0, 0, 0)^\top$.
    根据镜面反射矩阵的相关性质,设 单位向量 $\beta = \dfrac{A_{:, 1}}{||A_{:,1}||}$,$\epsilon = (1, 0, 0, 0)^\top$, 则可构造 $U_1 = E - \dfrac{2}{||\beta-\epsilon||^2}(\beta\beta^\top - \epsilon\beta^\top - \beta\epsilon^\top + \epsilon\epsilon^\top)$.
    计算过程较繁,这里给出化简结果: $$U_1 = \dfrac{1}{6-2\sqrt{6}}\times
    \left(\begin{array}{cccc}
        -4+2\sqrt{6}&2-\sqrt{6}&-2+\sqrt{6}&0\\
        2-\sqrt{6}&5-2\sqrt{6}&1&0\\
        -2+\sqrt{6}&1&5-2\sqrt{6}&0\\
        0&0&0&6-2\sqrt{6} \\
    \end{array}\right)$$
    计算得到 $$U_1A = \left(\begin{array}{cccc}
        \sqrt{6} & 0 & \sqrt{6}/2 & \sqrt{6}/2\\
        0 & -1 & -\frac{1}{2} & -\frac{1}{2} \\
        0 & -1 & -\frac{1}{2} & -\frac{1}{2} \\
        0 & -2 & -1 & -1 \\
    \end{array}\right)$$

    完全类似地,取 $\beta = \dfrac{1}{\sqrt 6}(-1, -1, -2)^\top$, $\epsilon = (1, 0, 0)^\top$, 计算得到 $3$ 阶的 $U_2$, 
    $$U_2=\dfrac{1}{6+\sqrt{6}}\times
    \left(\begin{array}{ccc}
        -1-\sqrt{6} & -1-\sqrt{6} & -2-2\sqrt{6}\\
        -1-\sqrt{6} & 5+\sqrt{6} & -2 \\
        -2-2\sqrt{6} & -2 & 2+\sqrt 6 \\
    \end{array}\right)$$

    那么 $U_2A_{1:3, 1:3} = \left(
        \begin{array}{ccc}
            \sqrt{6} & \frac{\sqrt{6}}{2} & \frac{\sqrt{6}}{2}\\
            0 & 0 & 0\\
            0 & 0 & 0\\
        \end{array}
    \right)$,已经是上三角矩阵。
    这就找到了 $A$ 的第二广义QR分解,$A=QR$, 其中 
    $R = \left(\begin{array}{cccc}
        \sqrt{6} & 0 & \sqrt{6}/2 & \sqrt{6}/2\\
        0 & \sqrt{6} & \sqrt{6}/2 & \sqrt{6}/2\\
        0 & 0 & 0 & 0\\
        0 & 0 & 0 & 0\\
    \end{array}\right) $, $Q = \left(\left(\begin{array}{cc}1&0\\0&U_2\end{array}\right)U_1\right)^{-1} = U_1^{-1}\left(\begin{array}{cc}1&0\\0&U_2^{-1}\end{array}\right)$.
    
    \item 2.2 - 习题3
    
    定理2.2.3可推广至非方阵的情形:

    任一矩阵 $A_{n\times m}$ 具有$QR$-分解,其中 $Q$ 是 $n$ 阶酉矩阵,而 $R$ 是 $n\times m$ 的上三角矩阵,且主对角线元素是非负实数。

    证明过程,可以完全仿照定理2.2.3,即不断运用引理2.2.2逐一消去矩阵 $A_{n\times m}$ 的列。矩阵是否是方阵,完全不影响引理2.2.2的使用。

    唯一的区别在于,当矩阵 $A_{n\times m}$ 消去 $\min\{n,m\}$ 列后,不能再继续像定理2.2.3证明中那样对剩余的子矩阵分块,所以至多只能消掉
    前$\min\{n,m\}$列中对角线以下的部分,但这不影响 $Q$ 是酉矩阵以及 $R$ 是非方形的上三角矩阵且对角线元素是非负实数。

\end{itemize}
\end{spacing}


\end{document}
