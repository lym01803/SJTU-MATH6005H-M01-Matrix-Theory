\documentclass{article}
\usepackage{ctex}
\usepackage{amsmath,amscd,amsbsy,amssymb,latexsym,url,bm,amsthm}
\usepackage{epsfig,graphicx,subfigure}
\usepackage{enumitem,balance,mathtools}
\usepackage{wrapfig}
\usepackage{mathrsfs, euscript}
\usepackage[usenames]{xcolor}
\usepackage{hyperref}
\usepackage{caption}
\usepackage{setspace}
%\usepackage{subcaption}
\usepackage{float}
\usepackage{listings}
%\usepackage{enumerate}
%\usepackage{algorithm}
%\usepackage{algorithmic}
%\usepackage[vlined,ruled,commentsnumbered,linesnumbered]{algorithm2e}
\usepackage[ruled,lined,boxed,linesnumbered]{algorithm2e}
\usepackage{tikz}

\newtheorem{theorem}{Theorem}[section]
\newtheorem{lemma}[theorem]{Lemma}
\newtheorem{proposition}[theorem]{Proposition}
\newtheorem{corollary}[theorem]{Corollary}
\newtheorem{exercise}{Exercise}[section]
\newtheorem*{solution}{Solution}

\renewcommand{\thefootnote}{\fnsymbol{footnote}}
\renewenvironment{solution}[1][Solution]{~\\ \textbf{#1.}}{~\\}

\newcommand{\prob}{\mathtt{Pr}}

\newcommand{\postscript}[2]
    {\setlength{\epsfxsize}{#2\hsize}
    \centerline{\epsfbox{#1}}}

\renewcommand{\baselinestretch}{1.0}
\SetKwFor{Function}{function}{:}{end}
\setlength{\oddsidemargin}{-0.365in}
\setlength{\evensidemargin}{-0.365in}
\setlength{\topmargin}{-0.3in}
\setlength{\headheight}{0in}
\setlength{\headsep}{0in}
\setlength{\textheight}{10.1in}
\setlength{\textwidth}{7in}

\title{矩阵理论 作业2}
\author{刘彦铭\ \ 学号:122033910081}
\date{编辑日期:\ \today}

\begin{document}

\maketitle

\begin{enumerate}
    \begin{spacing}{1.5}
    \item Page 16 习题 1
    
    课上已经讲过解法: 

    设$f(x), g(x)\in\mathbb{F}[x]$, $(f(x), g(x)) = 1 \Rightarrow \exists\ u(x), v(x)\in\mathbb{F}[x]$ 使得
    $f(x)\cdot u(x) + g(x)\cdot v(x) = 1$ .
    
    将上述多项式的 $x$ 代换为 $x^n$ 即得: $f(x^n)\cdot u(x^n) + g(x^n)\cdot v(x^n) = 1$ .

    容易验证 $u(x^n), v(x^n)\in\mathbb{F}[x]$, 这就证明了 $(f(x^n), g(x^n)) = 1$ .

    \item Page 16 习题 4
    
    \begin{enumerate}
        \item [(1)] 假设 $p(x)\in\mathbb{Q}[x]$ 也是满足 $p(\alpha)=0$ 的最低次的首一多项式.
        
        由于都是最低次的,所以有 $\deg m_\alpha = \deg p$. 
        
        作带余除法:存在多项式 $u, v\in\mathbb{Q}[x]$ 使得 $m_\alpha = u\cdot p + v$,  其中 $v=0$ 或者 $\deg v < \deg p$.

        注意到 $v(\alpha) = m_\alpha(\alpha) - u(\alpha)\cdot p(\alpha) = 0$, 所以有 $v=0$; 否则存在非零的多项式 $v\in\mathbb{Q}[x]$ 使得 $v(\alpha)=0$ 且 $\deg v < \deg p$,这与$p$最低次的假设矛盾。 

        所以 $m_\alpha = u\cdot p$. 因为 $\deg u = \deg m_\alpha - \deg p = 0$ 且 $m_\alpha, p$ 均首一, 所以 $u=1$
        
        因此 $m_\alpha = p$ , 这就说明了 $m_\alpha$ 的唯一性

        \item [(2)] 只需说明 $\{1,\alpha, \alpha^2, \cdots, \alpha^{m-1}\}$ 是 $\mathbb{Q}[\alpha]$ 的一组基:
        \begin{enumerate}
            \item [-] 线性无关:
                
            对任意的 $c_0, c_1, c_2, \cdots, c_{m-1}\in\mathbb{Q}$, 若 $f(\alpha) = \sum_{0\leq i < m} c_i\alpha^i = 0$, 
            由于 $\deg f = m-1 < \deg m_\alpha$, 所以由 $m_\alpha$ 的定义知 $f=0$, 即 $c_i=0,\forall\ 0\leq i < m$. 
            这就证明了 $\{\alpha^i\}, 0\leq i < m$ 的线性无关性。
            
            \item [-] 可表示性:
            
            对$\mathbb{Q}[\alpha]$上的任意一个元素 $\beta$, 由 $\mathbb{Q}[\alpha]$ 的生成方式可以知道,存在多项式 $f\in\mathbb{Q}[x]$, 
            使得 $\beta = f(\alpha)$

            考虑带余除法 $f = q\cdot m_\alpha + r$ 其中 $q, r\in\mathbb{Q}[x]$, $r=0$或$\deg r < \deg m_\alpha = m$. 

            于是 $\beta = f(\alpha) = q(\alpha)\cdot m_\alpha(\alpha) + r(\alpha) = r(\alpha) = \sum_{0\leq i < m} c_i \alpha^i$. 
            这就说明了 $\mathbb{Q}[\alpha]$ 上的任一元素都能由 $\{\alpha^i\}, 0\leq i < m$ 线性表示

        \end{enumerate}
    
    \end{enumerate}
    
    \end{spacing}

    \begin{spacing}{1.5}
    \item Page 16-17 习题5 (尝试做一下)
    \begin{enumerate}
        \item [(1)] 设 $p$ 是 $R$ 上的任意一个素元。对于任意的非零的 $p_1,p_2\in R$, 如果 $p = p_1p_2$, 那么有 $p \mid p_1p_2$. 由于 $p$ 是素元,所以 
        $p \mid p_1$ 或者 $p \mid p_2$ 。 不失一般性,假设 $p \mid p_1$ , 于是存在 $k\in R$, 使得 $p_1 = kp = kp_1p_2 = (kp_2)p_1$ (运用 $R$ 上
        的乘法交换律和结合律)。由于$R$是一个整环(这里略去证明)没有零因子,所以 $p_1 = (kp_2)p_1 \Rightarrow (kp_2 - 1)p_1 = 0 \Rightarrow kp_2 = 1$, 
        这就证明了 $p_2$ 是可逆元。

        \item [(2)] 
        
        \textbf{命题 1} 主理想整环$R$上的不可分解元都是素元。
        \begin{proof}
            假设 $c\in R$ 是一个不可分解元。对于主理想整环 $R$ 上的任意理想 $(a)$, 如果 $(c)\subset (a)\subset R$, 那么存在 $k\in R, c = ka$ .
            由于 $c$ 是不可分解的, 所以 $k$ 是可逆元 即 $(c) = (a)$ 或者 $a$ 是可逆元 即 $(a) = R$ 。这就验证了 $(c)$ 是一个极大理想。

            假设 不可分解元 $c$ 不是素元,那么存在非零的 $a, b\in R$ 使得 $c \mid ab$ 但 $c \nmid a$ , $c \nmid b$ .

            $c \nmid a \Rightarrow a \notin (c) \Rightarrow (c) \subset (a, c) \subset R$ 且 $(c) \ne (a, c)$. 其中 $(a, c)$ 表示 由 $a, c$ 生成的理想。
            由于 $(c)$ 是 极大的, 所以 $(a, c) = R$. 所以存在 $x, y\in R$ 使得 $ax + cy = 1$; 同理,存在 $n, m\in R$ 使得 $bn + cm = 1$. 
            稍做变换可以得到 $ab\cdot xn + c\cdot (y + m - ymc) = 1$. 说明 $ab$ 与 $c$ 生成的理想 $(ab, c) = (1) = R$. 但由于 $c \mid ab$ 所以 $(ab, c) = (c)$.
            这就导出了 $c$ 是单位元的平凡情形。所以$c$不是素元的假设不成立。
        \end{proof}
        
        \textbf{命题 2} 欧几里得整环都是主理想整环。
        \begin{proof}
            设 $I$ 是一欧几里得整环 $R$ 上的理想,设 $\phi : R\to \mathbb{N}$ 是定义在这一欧几里得环上的度量。可以从 $I$ 中选取出度量最小的元素 $a \in I$. 
            对于任意的 $b\in I$, 由于在欧几里得环上存在 $q, r\in R$ 使得 $b = q\cdot a + r$, 其中 $r=0$ 或者 $\phi(r) < \phi(a)$. 显然 $r = b - q\cdot a\in I$,
            所以 $\phi(r) < \phi(a)$ 不能成立,因此 $r=0$。 这就说明了 $I\subset (a)\subset I$, 即 $I = (a)$ 是可由 $a$ 生成的主理想。
        \end{proof}

        由命题1、2知,只需要验证 $R\in\{\mathbb{Z}, \mathbb{F}[x], \mathbb{Z}[i]\}$ 是欧几里得环。其中 $\mathbb{Z}, \mathbb{F}[x]$ 是十分常见的欧几里得环,这里略去验证,只
        验证 $\mathbb{Z}[i]$ 是欧几里得环:
        \begin{proof}
            定义度量 $\phi: \mathbb{Z}[i] \to \mathbb{N}$, $\phi(a) = |a|^2 = a\cdot \bar{a}$ .
            对于任意非零的 $a, b\in\mathbb{Z}[i]\subset \mathbb{Q}[i]$, $\dfrac{a}{b} = \dfrac{a\bar{b}}{b\bar{b}} = x + yi$, 其中 $x, y \in\mathbb{Q}$. 取距离 $x$, $y$ 
            最近的整数 $m$, $n$, 有  $|m - x| \leq 0.5, |n - y| \leq 0.5$. 构造 $q = m + ni \in\mathbb{Z}[i]$, $r = a - qb \in\mathbb{Z}[i]$, 使得 $a = qb + r$, 且其中 $r=0$ 
            或者 $\phi(r) = \phi(((x - m) + (y - n)i)\cdot b) = \phi((x - m) + (y - n) i)\cdot\phi(b) \leq 0.5 \cdot \phi(b) < \phi(b)$. 
        \end{proof}

        \item [(3)] $(2 + \sqrt {-5}) \nmid 3$ 但 $(2 + \sqrt{-5}) \mid 3 \times 3$. 所以 $2 +\sqrt{-5}$ 不是素元。同理 $2-\sqrt{-5}$, $3$ 都不是素元。
        考虑到在 $\mathbb{Z}[\sqrt{-5}]$ 上复数的模长的相关定义和性质仍然成立,故枚举$3, 2 + \sqrt{-5}, 2-\sqrt{-5}$可能的因子时,只需要考虑模长平方小于等于 $9$ 的,
        即只考虑 $a + b\sqrt{-5} \in\mathbb{Z}[\sqrt{-5}]$ 其中 $a,b \in \mathbb{Z}, a^2 + 5b^2 \leq 9$. 简单的穷举即可验证他们都是不可分解元。

    \end{enumerate}
    \end{spacing}

\end{enumerate}

\end{document}
