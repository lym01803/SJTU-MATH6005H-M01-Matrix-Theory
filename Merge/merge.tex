\documentclass{article}
\usepackage{ctex}
\usepackage{amsmath,amscd,amsbsy,amssymb,latexsym,url,bm,amsthm}
\usepackage{epsfig,graphicx,subfigure}
\usepackage{enumitem,balance,mathtools}
\usepackage{wrapfig}
\usepackage{mathrsfs, euscript}
\usepackage[usenames]{xcolor}
\usepackage{hyperref}
\usepackage{caption}
\usepackage{setspace}
%\usepackage{subcaption}
\usepackage{float}
\usepackage{listings}
%\usepackage{enumerate}
%\usepackage{algorithm}
%\usepackage{algorithmic}
%\usepackage[vlined,ruled,commentsnumbered,linesnumbered]{algorithm2e}
\usepackage[ruled,lined,boxed,linesnumbered]{algorithm2e}
\usepackage{tikz}

\newtheorem{theorem}{Theorem}[section]
\newtheorem{lemma}[theorem]{Lemma}
\newtheorem{proposition}[theorem]{Proposition}
\newtheorem{corollary}[theorem]{Corollary}
\newtheorem{exercise}{Exercise}[section]
\newtheorem*{solution}{Solution}

\renewcommand{\thefootnote}{\fnsymbol{footnote}}
\renewenvironment{solution}[1][Solution]{~\\ \textbf{#1.}}{~\\}

\newcommand{\prob}{\mathtt{Pr}}

\newcommand{\postscript}[2]
{\setlength{\epsfxsize}{#2\hsize}
\centerline{\epsfbox{#1}}}

\renewcommand{\baselinestretch}{1.0}
\SetKwFor{Function}{function}{:}{end}
\setlength{\oddsidemargin}{-0.365in}
\setlength{\evensidemargin}{-0.365in}
\setlength{\topmargin}{-0.3in}
\setlength{\headheight}{0in}
\setlength{\headsep}{0in}
\setlength{\textheight}{10.1in}
\setlength{\textwidth}{7in}

\title{矩阵理论\ 作业汇总} 
\author{刘彦铭\ \ ID: 122033910081}
\date{Last Edited:\ \today}

\begin{document}

\maketitle

\begin{spacing}{1.5}

\begin{enumerate}
    \item Page 5 习题1
    
    从镜面反射变换的几何意义来看:
        \begin{figure}[htb]
            \centering 
            \begin{tikzpicture}
                \node (C) at (4, 4) {$C$};
                \node (A) at (8, 2) {$A$};
                \node (B) at (0, 2) {$B$};
                \node (O) at (4, 0) {$O$};
                \node (M) at (4, 2) {$M$};

                \draw [->] (A) -- node[right]{$(1, 1, 1, 1)^\top$} (O);
                \draw [->] (O) -- node[left]{$(1, 0, 0, \sqrt3)^\top$} (B);
                \draw [->] (O) -- (C);
            \end{tikzpicture}
        \end{figure}
    
    镜面方向上的 $\alpha$ 与 $\overrightarrow{OC} = \overrightarrow{OB} + \overrightarrow{OA} = \overrightarrow{OB} - \overrightarrow{AO} = (0, -1, -1, \sqrt3-1)^\top$ 同方向
    
    故单位向量 $\alpha=\dfrac{\overrightarrow{OC}}{|\overrightarrow{OC}|}=\dfrac{1}{\sqrt{6-2\sqrt3}}(0, -1, -1, \sqrt3 - 1)^\top$

    所以 
    $$B=E-2\alpha\alpha^T=\dfrac{1}{3 - \sqrt 3}\left[\begin{array}{cccc}
        3 - \sqrt 3 & 0 & 0 & 0 \\
        0 & 2 - \sqrt 3 & -1 & \sqrt 3 - 1 \\
        0 & -1 & 2 -\sqrt 3 & \sqrt 3 - 1 \\
        0 & \sqrt 3 - 1 & \sqrt 3 - 1 & \sqrt 3 - 1    
    \end{array}\right]$$

    \item Page 11 习题1 
    
    可将该矩阵分解为行变换矩阵和分块对角阵的乘积:
    \begin{equation*}
        \left[
            \begin{array}{ccc}
                0 & B & 0 \\
                E & 0 & 0 \\
                0 & 0 & A \\
            \end{array}
        \right]
        = 
        \left[
            \begin{array}{ccc}
                0 & E & 0 \\
                E & 0 & 0 \\
                0 & 0 & E \\
            \end{array}
        \right]
        \times
        \left[
            \begin{array}{ccc}
                E & 0 & 0 \\
                0 & B & 0 \\
                0 & 0 & A \\
            \end{array}
        \right]
    \end{equation*}
    故其逆矩阵为 
    \begin{equation*}
        \left[
            \begin{array}{ccc}
                0 & B & 0 \\
                E & 0 & 0 \\
                0 & 0 & A \\
            \end{array}
        \right]^{-1}
        =
        \left[
            \begin{array}{ccc}
                E & 0 & 0 \\
                0 & B^{-1} & 0 \\
                0 & 0 & A^{-1} \\
            \end{array}
        \right]
        \times
        \left[
            \begin{array}{ccc}
                0 & E & 0 \\
                E & 0 & 0 \\
                0 & 0 & E \\
            \end{array}
        \right]
        =
        \left[
            \begin{array}{ccc}
                0 & E & 0 \\
                B^{-1} & 0 & 0 \\
                0 & 0 & A^{-1} \\
            \end{array}
        \right]
    \end{equation*}
    行列式:
    \begin{equation*}
        \left|
            \begin{array}{ccc}
                0 & B & 0 \\
                E & 0 & 0 \\
                0 & 0 & A \\
            \end{array}
        \right|
        = 
        \left|
            \begin{array}{ccc}
                0 & E & 0 \\
                E & 0 & 0 \\
                0 & 0 & E \\
            \end{array}
        \right|
        \times
        \left|
            \begin{array}{ccc}
                E & 0 & 0 \\
                0 & B & 0 \\
                0 & 0 & A \\
            \end{array}
        \right|
        = (-1)^{n}\cdot \det(A)\cdot\det(B)
    \end{equation*}
    \item Page 11 习题3
    
    注意到对该方阵进行行变换可以得到 
    \begin{equation*}
        \left[
            \begin{array}{cc}
                E & 0 \\
                -C & E \\
            \end{array}
        \right]
        \times
        \left[
            \begin{array}{cc}
                A^{-1} & 0 \\
                0 & E_m \\
            \end{array}
        \right]
        \times
        \left[
            \begin{array}{cc}
                A & B \\
                C & D_m
            \end{array}
        \right]
        = 
        \left[
            \begin{array}{cc}
                E & A^{-1}B \\
                0 & D_m - CA^{-1}B \\
            \end{array}
        \right]
    \end{equation*}
    上式中左边两个行变换矩阵均满秩,故 $H=\left[\begin{array}{cc}A&B\\C&D_m\\\end{array}\right]$ 可逆的充要条件为
    \begin{equation*}
        \left|
            \begin{array}{cc}
                E & A^{-1}B \\
                0 & D_m - CA^{-1}B \\
            \end{array}
        \right|
        = \det(D_m - CA^{-1}B)\ne 0
    \end{equation*}
    (1) 由上述讨论知,充要条件是 $m$ 阶方阵 $D_m-CA^{-1}B$ 可逆;
    
    (2) 简单起见,令 $D^\prime=D_m-CA^{-1}B$, $H$可逆时, $D^{\prime-1}$ 唯一存在。
    注意到,运用行变换有:

    \begin{equation*}
        \left[
            \begin{array}{cc}
                E & -A^{-1}B \\
                0 & E \\
            \end{array}
        \right]
        \times
        \left[
            \begin{array}{cc}
                E & 0 \\
                0 & D^{\prime-1} \\
            \end{array}
        \right] 
        \times 
        \left[
            \begin{array}{cc}
                E & A^{-1}B \\
                0 & D^\prime \\
            \end{array}
        \right]
        = E_{n+m}
    \end{equation*}
    所以 
    \begin{spacing}{1.5}
    \begin{equation*}
        \begin{array}{ll}
            H^{-1} & = 
            \left[
                \begin{array}{cc}
                    E & -A^{-1}B \\
                    0 & E \\
                \end{array}
            \right]
            \times
            \left[
                \begin{array}{cc}
                    E & 0 \\
                    0 & D^{\prime-1} \\
                \end{array}
            \right] 
            \times
            \left[
                \begin{array}{cc}
                    E & 0 \\
                    -C & E \\
                \end{array}
            \right]
            \times
            \left[
                \begin{array}{cc}
                    A^{-1} & 0 \\
                    0 & E_m \\
                \end{array}
            \right] \\
            & = 
            \left[
                \begin{array}{cc}
                    A^{-1} + A^{-1}B(D_m-CA^{-1}B)^{-1}CA^{-1} & -A^{-1}B(D_m-CA^{-1}B)^{-1} \\
                    -(D_m-CA^{-1}B)^{-1}CA^{-1} & (D_m-CA^{-1}B)^{-1}\\
                \end{array}
            \right]
        \end{array}
    \end{equation*}
    \end{spacing}
    非常丑,但验证了一下应该是对的。左上角似乎和Woodbury公式的形式是一致的。
    
    \item Page 12 习题6
    
    考虑当 $A\in \mathbb{R}^{n\times n}$ 时,对任意列向量 $x^\prime=\left[x^\top; y\right]^\top\in\mathbb{R}^{(n+1)\times 1}$, 其中 $x\in\mathbb{R}^{n\times 1}, y\in\mathbb{R}$:

    计算得到

    \begin{spacing}{1.5}
        
    \begin{equation*}
        \begin{array}{ll}
            f(x^\prime)=x^{\prime\top}\left[\begin{array}{cc}A&k\alpha\\k\alpha^\top&1\\\end{array}\right]x^\prime 
            &= 
            \left[x^\top; y\right]\left[\begin{array}{cc}A&k\alpha\\k\alpha^\top&1\\\end{array}\right]\left[\begin{array}{c}x\\y\end{array}\right]
            \\
            &= x^\top Ax + 2ky(x^\top\alpha)+y^2
            \\
            &= \left(y + k(x^\top\alpha)\right)^2 + x^\top Ax - k^2(x^\top\alpha x^\top\alpha)\\
            &= \left(y + k(x^\top\alpha)\right)^2 + x^\top (A - k^2\alpha\alpha^\top) x

        \end{array}
    \end{equation*}

    \end{spacing}

    \begin{itemize}
        \item [-] 若 $A-k^2\alpha\alpha^\top$ 是正定矩阵,那么 $\left[\begin{array}{cc}A&k\alpha\\k\alpha^\top&1\\\end{array}\right]$ 正定。因为:
        \begin{itemize}
            \item [--] 一方面 $f(x^\prime)\geq 0$ 恒成立;
            \item [--] 另一方面, $f(x^\prime)=0$ 可以推出 $x=0, y=0, x^\prime=0$.
        \end{itemize}
        \item [-] 若 $A-k^2\alpha\alpha^\top$ 是半正定矩阵,那么 $\left[\begin{array}{cc}A&k\alpha\\k\alpha^\top&1\\\end{array}\right]$ 半正定。因为:
        \begin{itemize}
            \item [--] 一方面 $f(x^\prime)\geq 0$ 恒成立;
            \item [--] 另一方面, 由于存在$x\ne 0$ 使得 $x^\top (A-k^2\alpha\alpha^\top)x =0$, 取 $y=-k(x^\top\alpha)$ 即得到非零的 $x^\prime$ 使得 $f(x^\prime)=0$.
        \end{itemize}
        \item [-] 若 $A-k^2\alpha\alpha^\top$ 是不定矩阵,那么 $\left[\begin{array}{cc}A&k\alpha\\k\alpha^\top&1\\\end{array}\right]$ 也是不定的。因为:
        此时 $A-k^2\alpha\alpha^\top$ 存在小于 $0$ 的特征值,取 $x$ 为其对应的特征向量,取 $y=-k(x^\top\alpha)$ 即构造得到 $x^\prime$, $f(x^\prime) < 0$.
    \end{itemize}


    \item Page 13 习题7
    
    方便起见,令 $C=AB\in\mathbb{F}^{m\times r}$, 直接计算验证:对于任一 $1\leq k\leq r$,
    \begin{equation*}
        \begin{array}{ll}
            \sum_{1\leq i\leq m} C_{ik} & = \sum_{1\leq i\leq m} \sum_{1\leq j\leq n} A_{ij}\times B_{jk} \\
            & = \sum_{1\leq j\leq n} \sum_{1\leq i\leq m} A_{ij}\times B_{jk} \\
            & = \sum_{1\leq j\leq n} B_{jk}\times\left(\sum_{1\leq i\leq m} A_{ij}\right) \\
            & = \sum_{1\leq j\leq n} B_{jk} \times a \\ 
            & = ab
        \end{array}
    \end{equation*}

    \item Page 16 习题 1
    
    课上已经讲过解法: 

    设$f(x), g(x)\in\mathbb{F}[x]$, $(f(x), g(x)) = 1 \Rightarrow \exists\ u(x), v(x)\in\mathbb{F}[x]$ 使得
    $f(x)\cdot u(x) + g(x)\cdot v(x) = 1$ .
    
    将上述多项式的 $x$ 代换为 $x^n$ 即得: $f(x^n)\cdot u(x^n) + g(x^n)\cdot v(x^n) = 1$ .

    容易验证 $u(x^n), v(x^n)\in\mathbb{F}[x]$, 这就证明了 $(f(x^n), g(x^n)) = 1$ .

    \item Page 16 习题 4
    
    \begin{enumerate}
        \item [(1)] 假设 $p(x)\in\mathbb{Q}[x]$ 也是满足 $p(\alpha)=0$ 的最低次的首一多项式.
        
        由于都是最低次的,所以有 $\deg m_\alpha = \deg p$. 
        
        作带余除法:存在多项式 $u, v\in\mathbb{Q}[x]$ 使得 $m_\alpha = u\cdot p + v$,  其中 $v=0$ 或者 $\deg v < \deg p$.

        注意到 $v(\alpha) = m_\alpha(\alpha) - u(\alpha)\cdot p(\alpha) = 0$, 所以有 $v=0$; 否则存在非零的多项式 $v\in\mathbb{Q}[x]$ 使得 $v(\alpha)=0$ 且 $\deg v < \deg p$,这与$p$最低次的假设矛盾。 

        所以 $m_\alpha = u\cdot p$. 因为 $\deg u = \deg m_\alpha - \deg p = 0$ 且 $m_\alpha, p$ 均首一, 所以 $u=1$
        
        因此 $m_\alpha = p$ , 这就说明了 $m_\alpha$ 的唯一性

        \item [(2)] 只需说明 $\{1,\alpha, \alpha^2, \cdots, \alpha^{m-1}\}$ 是 $\mathbb{Q}[\alpha]$ 的一组基:
        \begin{enumerate}
            \item [-] 线性无关:
                
            对任意的 $c_0, c_1, c_2, \cdots, c_{m-1}\in\mathbb{Q}$, 若 $f(\alpha) = \sum_{0\leq i < m} c_i\alpha^i = 0$, 
            由于 $\deg f = m-1 < \deg m_\alpha$, 所以由 $m_\alpha$ 的定义知 $f=0$, 即 $c_i=0,\forall\ 0\leq i < m$. 
            这就证明了 $\{\alpha^i\}, 0\leq i < m$ 的线性无关性。
            
            \item [-] 可表示性:
            
            对$\mathbb{Q}[\alpha]$上的任意一个元素 $\beta$, 由 $\mathbb{Q}[\alpha]$ 的生成方式可以知道,存在多项式 $f\in\mathbb{Q}[x]$, 
            使得 $\beta = f(\alpha)$

            考虑带余除法 $f = q\cdot m_\alpha + r$ 其中 $q, r\in\mathbb{Q}[x]$, $r=0$或$\deg r < \deg m_\alpha = m$. 

            于是 $\beta = f(\alpha) = q(\alpha)\cdot m_\alpha(\alpha) + r(\alpha) = r(\alpha) = \sum_{0\leq i < m} c_i \alpha^i$. 
            这就说明了 $\mathbb{Q}[\alpha]$ 上的任一元素都能由 $\{\alpha^i\}, 0\leq i < m$ 线性表示

        \end{enumerate}
    
    \end{enumerate}
    
    \item Page 16-17 习题5 (尝试做一下)
    \begin{enumerate}
        \item [(1)] 设 $p$ 是 $R$ 上的任意一个素元。对于任意的非零的 $p_1,p_2\in R$, 如果 $p = p_1p_2$, 那么有 $p \mid p_1p_2$. 由于 $p$ 是素元,所以 
        $p \mid p_1$ 或者 $p \mid p_2$ 。 不失一般性,假设 $p \mid p_1$ , 于是存在 $k\in R$, 使得 $p_1 = kp = kp_1p_2 = (kp_2)p_1$ (运用 $R$ 上
        的乘法交换律和结合律)。由于$R$是一个整环(这里略去证明)没有零因子,所以 $p_1 = (kp_2)p_1 \Rightarrow (kp_2 - 1)p_1 = 0 \Rightarrow kp_2 = 1$, 
        这就证明了 $p_2$ 是可逆元。

        \item [(2)] 
        
        \textbf{命题 1} 主理想整环$R$上的不可分解元都是素元。
        \begin{proof}
            假设 $c\in R$ 是一个不可分解元。对于主理想整环 $R$ 上的任意理想 $(a)$, 如果 $(c)\subset (a)\subset R$, 那么存在 $k\in R, c = ka$ .
            由于 $c$ 是不可分解的, 所以 $k$ 是可逆元 即 $(c) = (a)$ 或者 $a$ 是可逆元 即 $(a) = R$ 。这就验证了 $(c)$ 是一个极大理想。

            假设 不可分解元 $c$ 不是素元,那么存在非零的 $a, b\in R$ 使得 $c \mid ab$ 但 $c \nmid a$ , $c \nmid b$ .

            $c \nmid a \Rightarrow a \notin (c) \Rightarrow (c) \subset (a, c) \subset R$ 且 $(c) \ne (a, c)$. 其中 $(a, c)$ 表示 由 $a, c$ 生成的理想。
            由于 $(c)$ 是 极大的, 所以 $(a, c) = R$. 所以存在 $x, y\in R$ 使得 $ax + cy = 1$; 同理,存在 $n, m\in R$ 使得 $bn + cm = 1$. 
            稍做变换可以得到 $ab\cdot xn + c\cdot (y + m - ymc) = 1$. 说明 $ab$ 与 $c$ 生成的理想 $(ab, c) = (1) = R$. 但由于 $c \mid ab$ 所以 $(ab, c) = (c)$.
            这就导出了 $c$ 是单位元的平凡情形。所以$c$不是素元的假设不成立。
        \end{proof}
        
        \textbf{命题 2} 欧几里得整环都是主理想整环。
        \begin{proof}
            设 $I$ 是一欧几里得整环 $R$ 上的理想,设 $\phi : R\to \mathbb{N}$ 是定义在这一欧几里得环上的度量。可以从 $I$ 中选取出度量最小的元素 $a \in I$. 
            对于任意的 $b\in I$, 由于在欧几里得环上存在 $q, r\in R$ 使得 $b = q\cdot a + r$, 其中 $r=0$ 或者 $\phi(r) < \phi(a)$. 显然 $r = b - q\cdot a\in I$,
            所以 $\phi(r) < \phi(a)$ 不能成立,因此 $r=0$。 这就说明了 $I\subset (a)\subset I$, 即 $I = (a)$ 是可由 $a$ 生成的主理想。
        \end{proof}

        由命题1、2知,只需要验证 $R\in\{\mathbb{Z}, \mathbb{F}[x], \mathbb{Z}[i]\}$ 是欧几里得环。其中 $\mathbb{Z}, \mathbb{F}[x]$ 是十分常见的欧几里得环,这里略去验证,只
        验证 $\mathbb{Z}[i]$ 是欧几里得环:
        \begin{proof}
            定义度量 $\phi: \mathbb{Z}[i] \to \mathbb{N}$, $\phi(a) = |a|^2 = a\cdot \bar{a}$ .
            对于任意非零的 $a, b\in\mathbb{Z}[i]\subset \mathbb{Q}[i]$, $\dfrac{a}{b} = \dfrac{a\bar{b}}{b\bar{b}} = x + yi$, 其中 $x, y \in\mathbb{Q}$. 取距离 $x$, $y$ 
            最近的整数 $m$, $n$, 有  $|m - x| \leq 0.5, |n - y| \leq 0.5$. 构造 $q = m + ni \in\mathbb{Z}[i]$, $r = a - qb \in\mathbb{Z}[i]$, 使得 $a = qb + r$, 且其中 $r=0$ 
            或者 $\phi(r) = \phi(((x - m) + (y - n)i)\cdot b) = \phi((x - m) + (y - n) i)\cdot\phi(b) \leq 0.5 \cdot \phi(b) < \phi(b)$. 
        \end{proof}

        \item [(3)] $(2 + \sqrt {-5}) \nmid 3$ 但 $(2 + \sqrt{-5}) \mid 3 \times 3$. 所以 $2 +\sqrt{-5}$ 不是素元。同理 $2-\sqrt{-5}$, $3$ 都不是素元。
        考虑到在 $\mathbb{Z}[\sqrt{-5}]$ 上复数的模长的相关定义和性质仍然成立,故枚举$3, 2 + \sqrt{-5}, 2-\sqrt{-5}$可能的因子时,只需要考虑模长平方小于等于 $9$ 的,
        即只考虑 $a + b\sqrt{-5} \in\mathbb{Z}[\sqrt{-5}]$ 其中 $a,b \in \mathbb{Z}, a^2 + 5b^2 \leq 9$. 简单的穷举即可验证他们都是不可分解元。
    \end{enumerate}

    \item 2.1 - 习题3
    
    对矩阵 $A$ 模拟高斯消元过程可以知道,需要将原第2行置于第1行,将原第1行置于第2行,于是可以得到可行的置换矩阵 $P = \left(\begin{array}{ccc}0&1&0\\1&0&0\\0&0&1\end{array}\right)$
    
    对 $PA$ 模拟高斯消元过程可得 :
    $$
    \begin{array}{ll}
        PA = \left(\begin{array}{ccc}1&2&3\\0&1&1\\1&2&4\end{array}\right)
           &= \left(\begin{array}{ccc}1&0&0\\0&1&0\\-1&0&1\end{array}\right)^{-1}\times 
              \left(\begin{array}{ccc}1&2&3\\0&1&1\\0&0&1\end{array}\right)\\
           &= \left(\begin{array}{ccc}1&0&0\\0&1&0\\1&0&1\end{array}\right)\times 
              \left(\begin{array}{ccc}1&2&3\\0&1&1\\0&0&1\end{array}\right)\\
    \end{array}
    $$

    于是 $L = \left(\begin{array}{ccc}1&0&0\\0&1&0\\1&0&1\end{array}\right)$, $U=\left(\begin{array}{ccc}1&2&3\\0&1&1\\0&0&1\end{array}\right)$.

    \item 2.2 - 习题1
    
    交换1,3两列,并选取前两列作为列向量的极大无关组。有:$P = \left(\begin{array}{cccc}0&0&1&0\\0&1&0&0\\1&0&0&0\\0&0&0&1\end{array}\right)$
    $AP = \left(\begin{array}{cccc}1&0&2&1\\-1&-1&-1&-1\\0&-1&1&0\\-1&-2&0&-1\end{array}\right)$
    对前两列做Schmidt正交化,再单位化,得到:

    $$
    AP = \left(\begin{array}{cc}
        \dfrac{1}{\sqrt 3} & \dfrac{1}{\sqrt 3}\\
        -\dfrac{1}{\sqrt 3} & 0\\
        0 & \dfrac{1}{\sqrt 3}\\
        -\dfrac{1}{\sqrt 3} & \dfrac{1}{\sqrt 3}\\
    \end{array}\right)
    \times 
    \left(\begin{array}{cccc}
        \sqrt{3} & \sqrt{3} & \sqrt{3} & \sqrt{3}\\
        0 & -\sqrt{3} & \sqrt{3} & 0\\
    \end{array}\right)
    = QR
    $$

    \item 2.2 - 习题2
    
    $A = \left(\begin{array}{cccc}2&0&1&1\\-1&-1&-1&-1\\1&-1&0&0\\0&-2&-1&-1\end{array}\right)$.
    对于 $A$ 的第一列, $A_{:,1} = (-2, -1, 1, 0)^\top$, 我们希望构造 酉矩阵 $U_1$ 使得 $U_1 A_{:,1} = (||A_{:,1}||, 0, 0, 0)^\top$.
    根据镜面反射矩阵的相关性质,设 单位向量 $\beta = \dfrac{A_{:, 1}}{||A_{:,1}||}$,$\epsilon = (1, 0, 0, 0)^\top$, 则可构造 $U_1 = E - \dfrac{2}{||\beta-\epsilon||^2}(\beta\beta^\top - \epsilon\beta^\top - \beta\epsilon^\top + \epsilon\epsilon^\top)$.
    计算过程较繁,这里给出化简结果: $$U_1 = \dfrac{1}{6-2\sqrt{6}}\times
    \left(\begin{array}{cccc}
        -4+2\sqrt{6}&2-\sqrt{6}&-2+\sqrt{6}&0\\
        2-\sqrt{6}&5-2\sqrt{6}&1&0\\
        -2+\sqrt{6}&1&5-2\sqrt{6}&0\\
        0&0&0&6-2\sqrt{6} \\
    \end{array}\right)$$
    计算得到 $$U_1A = \left(\begin{array}{cccc}
        \sqrt{6} & 0 & \sqrt{6}/2 & \sqrt{6}/2\\
        0 & -1 & -\frac{1}{2} & -\frac{1}{2} \\
        0 & -1 & -\frac{1}{2} & -\frac{1}{2} \\
        0 & -2 & -1 & -1 \\
    \end{array}\right)$$

    完全类似地,取 $\beta = \dfrac{1}{\sqrt 6}(-1, -1, -2)^\top$, $\epsilon = (1, 0, 0)^\top$, 计算得到 $3$ 阶的 $U_2$, 
    $$U_2=\dfrac{1}{6+\sqrt{6}}\times
    \left(\begin{array}{ccc}
        -1-\sqrt{6} & -1-\sqrt{6} & -2-2\sqrt{6}\\
        -1-\sqrt{6} & 5+\sqrt{6} & -2 \\
        -2-2\sqrt{6} & -2 & 2+\sqrt 6 \\
    \end{array}\right)$$

    那么 $U_2A_{1:3, 1:3} = \left(
        \begin{array}{ccc}
            \sqrt{6} & \frac{\sqrt{6}}{2} & \frac{\sqrt{6}}{2}\\
            0 & 0 & 0\\
            0 & 0 & 0\\
        \end{array}
    \right)$,已经是上三角矩阵。
    这就找到了 $A$ 的第二广义QR分解,$A=QR$, 其中 
    $R = \left(\begin{array}{cccc}
        \sqrt{6} & 0 & \sqrt{6}/2 & \sqrt{6}/2\\
        0 & \sqrt{6} & \sqrt{6}/2 & \sqrt{6}/2\\
        0 & 0 & 0 & 0\\
        0 & 0 & 0 & 0\\
    \end{array}\right) $, $Q = \left(\left(\begin{array}{cc}1&0\\0&U_2\end{array}\right)U_1\right)^{-1} = U_1^{-1}\left(\begin{array}{cc}1&0\\0&U_2^{-1}\end{array}\right)$.
    
    \item 2.2 - 习题3
    
    定理2.2.3可推广至非方阵的情形:

    任一矩阵 $A_{n\times m}$ 具有$QR$-分解,其中 $Q$ 是 $n$ 阶酉矩阵,而 $R$ 是 $n\times m$ 的上三角矩阵,且主对角线元素是非负实数。

    证明过程,可以完全仿照定理2.2.3,即不断运用引理2.2.2逐一消去矩阵 $A_{n\times m}$ 的列。矩阵是否是方阵,完全不影响引理2.2.2的使用。

    唯一的区别在于,当矩阵 $A_{n\times m}$ 消去 $\min\{n,m\}$ 列后,不能再继续像定理2.2.3证明中那样对剩余的子矩阵分块,所以至多只能消掉
    前$\min\{n,m\}$列中对角线以下的部分,但这不影响 $Q$ 是酉矩阵以及 $R$ 是非方形的上三角矩阵且对角线元素是非负实数。

    \item 2.3 - 习题2
    
    \begin{itemize}
        \item [$\Leftarrow$]
            存在实正交矩阵$P\in\mathbb{R}^{n\times n}$使得 $P^\top BP = \left[\begin{array}{cc}-1&0\\0&E_{n-1}\end{array}\right]$, 所以 
            $B = P\left(E_n - 2\cdot\left[\begin{array}{cc}1&0\\0&O_{n-1}\end{array}\right]\right)P^\top$ 展开即可得到:$B = E - 2p_1p_1^\top$, 其中 $p_1$ 是
            实正交矩阵的第$1$列,是单位向量,这就验证了 $B$ 是镜面反射矩阵。
        \item [$\Rightarrow$] 由于 $B$ 是镜面反射矩阵,所以存在某个单位长度的 $\delta\in\mathbb{C}^{n\times 1}$ 使得 $B = E - 2\delta\delta^\star$。
        
        由于 $B$ 是实方阵,所以 $\delta\delta^\star$ 
        也必须是实矩阵,即 $\delta_i\delta_j\in\mathbb{R},\forall 1\leq i,j \leq n$,这要求$\delta$中的各分量的辐角彼此相差$\pi$的整数倍,所以$\delta$可以拆分作 $\delta=\mathtt{e}^{i\theta}\cdot \delta^\prime$,
        其中$\delta^\prime$是实单位向量。

        仿照引理2.2.2的推导可知,存在实正交矩阵 $P$ 使得 $P^\top\delta^\prime = \left[1, 0, 0, \cdots, 0\right]^\top$. (注:引理2.2.2是针对复数域的,但由于这里的$\delta^\prime$是实向量,故可以按照完全相同的方法构造出实的镜面反射矩阵$P$)
        
        故 $P^\top B P = P^\top(E-2\delta\delta^\star)P = P^\top \left(E - 2(\mathtt{e}^{i\theta}\delta^\prime)(\mathtt{e}^{-i\theta}{\delta^\prime}^\top)\right)P = E - 2(P^\top\delta^\prime)(P^\top\delta^\prime)^\top = \mathtt{diag}\{-1, E\}$

    \end{itemize}

    \item 2.4 - 习题5
    
    \begin{itemize}
        \item [(1)]由于 $A$ 是正规矩阵,根据正规矩阵基本定理,$A$酉相似于对角阵,即存在酉矩阵$U$, $U^\star A U = \mathtt{diag}\{\lambda_1, \lambda_2, \cdots, \lambda_n\}$, 且根据 Schur 引理的推导过程知,对角线上元素$\lambda_i$为 $A$ 的特征值,在本题中他们两两互异。

            所以 $A = U\Lambda U^\star$, 其中 $\Lambda = \mathtt{diag}\{\lambda_1, \lambda_2, \cdots, \lambda_n\}$. $AB=BA \Rightarrow U\Lambda U^\star B = B U\Lambda U^\star \Rightarrow$ 
            
            $\Lambda U^\star B U = U^\star B U \Lambda$. 考虑左右两边的 $i,j$ 位置元素, 可得 $\lambda_i (U^\star B U)_{ij} = \lambda_j (U^\star B U)_{ij}$, 再由$\lambda_i$两两互异可知,$(U^\star B U)$对角线以外的元素均为0,即 $B$ 酉相似于对角阵,所以 $B$ 是正规矩阵。
        \item[(2)] 由(1)的推导可知, 任意$B\in\mathbb{C}(A), AB=BA$, 都有 $B = U\Lambda_B U^\star$, 且对不同的$B$, 对应的酉矩阵$U$都相同,都由 $A$ 决定。$U = (u_1, u_2, \cdots, u_n)$, 则 $u_1u_1^\star, u_2u_2^\star,\cdots, u_nu_n^\star$ 构成 $\mathbb{C}(A)$
        的一组基。因为对于任意的 $B$, $B= \sum_i {\Lambda_B}_i u_iu_i^\star$, 且$u_iu_i^\star$之间彼此正交, ${\Lambda_B}_{i}\in\mathbb{F}$。所以 $\mathbb{C}(A)$ 是数域 $\mathbb{F}$ 上的 $n$ 维线性空间。
        加法、数乘的封闭性显然。乘法的封闭性由 $(u_iu_i^\star)(u_ju_j^\star) = \delta_{ij} u_iu_i^\star$ 保证。
        
    \end{itemize}
    
    

    \item 2.4 - 习题6
    
    先证明一个性质 $tr(AB) = tr(BA)$. $tr(AB) = \sum_{i} (AB)_{ii} = \sum_{i} \sum_{k} A_{ik}B_{ki} = \sum_{k}\sum_{i} B_{ki}A_{ik} = \sum_k (BA)_{kk} = tr(BA)$.

    \begin{itemize}
        \item [(1)] $A_n$ 是Hermite矩阵 $\Leftrightarrow$ $tr(A^2) = tr(A^\star A)$.
        \begin{itemize}
            \item [$\Rightarrow$:] $A^\star = A$, 故 $tr(AA) = tr(A^\star A)$ 显然成立。
            \item [$\Leftarrow$:] 据Schur引理,存在酉矩阵 $Q$ 使得 $QAQ^\star = U$, 其中 $U$ 是上三角矩阵, $A = Q^\star U Q$.
            
            于是有 $tr(AA) = tr(Q^\star U Q Q^\star U Q) = tr(Q^\star U U Q) = tr(UUQQ^\star) = tr(UU)$.
            
            类似地,$tr(A^\star A) = tr(Q^\star U^\star QQ^\star U Q) = tr(Q^\star U^\star U Q) = tr(U^\star U Q Q^\star) = tr(U^\star U)$. 
        
            注意到 $U$ 是上三角矩阵,$U^\star$ 是下三角矩阵,$tr(UU) = \sum_{k} U_{kk}U_{kk}$. 而 $tr(U^\star U) = \sum_{i,j}\overline{U_{ij}}U_{ij}$
            
            由于 $tr(U^\star U)$ 得到的是实数,现考虑 $tr(UU) = \sum_{k} U_{kk}U_{kk}$ 的实部。
            容易证明 $\mathtt{Re}(U_{kk}U_{kk}) = \mathtt{Re}(U_{kk})^2 - \mathtt{Im}(U_{kk})^2 \leq \mathtt{Re}(U_{kk})^2 + \mathtt{Im}(U_{kk})^2 = \overline{U_{kk}}U_{kk}$.
            当且仅当 $U_{kk}$ 为实数时取等号。所以 由 $tr(UU) = \sum_{k} U_{kk}U_{kk} = \sum_{i,j}\overline{U_{ij}}U_{ij} = tr(U^\star U)$ 推出 $U_{kk}$ 都是实数,且 $U_{ij}=0, \forall i\ne j$.

            所以 $U$ 是实对角矩阵,$U^\star = U$. 所以 $A^\star = Q^\star U^\star Q = Q^\star U Q = A$, $A$ 是 Hermite矩阵。

        \end{itemize}
        \item [(2)] $A, B$ 都是 Hermite矩阵, $AB = BA \Leftrightarrow tr((AB)^2) = tr(A^2B^2)$.
        \begin{itemize}
            \item [$\Rightarrow$:] $AB = BA \Rightarrow (AB)^\star = B^\star A^\star = BA = AB$, 所以 $AB$ 是 Hermite矩阵. 由 (1) 知,$tr((AB)^2) = tr((AB)^\star AB) = tr(BAAB) = tr(AABB) = tr(A^2B^2)$.
            \item [$\Leftarrow$:] $tr((AB)^2) = tr(A^2B^2) \Rightarrow tr(ABAB) = tr(AABB) \Rightarrow tr(ABAB) = tr(BAAB) \Rightarrow tr(ABAB) = tr((AB)^\star AB)$. 由 (1) 知 $AB$ 是 Hermite矩阵,
            所以 $BA = B^\star A^\star = (AB)^\star = AB$. 
        \end{itemize}
    \end{itemize}

    \item [2.5 习题2]
        
        \begin{itemize}
            \item [$\Rightarrow$:]
            
            $A$ 与 $B$ 酉等价,故而存在酉矩阵 $U_m, V_n$ 使得 $U_m A V_n = B$, $V_n^\star A^\star U_m^\star = B^\star$. 所以 存在酉矩阵 
            $U_1 = \left(\begin{array}{cc}0&U_m\\V_n^\star&0\end{array}\right)$, 使得 
            $U_1 \left(\begin{array}{cc}0&A\\A^\star&0\end{array}\right) U_1^\star = \left(\begin{array}{cc}0&B\\B^\star&0\end{array}\right)$.
            故这两分块矩阵酉相似。

            \item [$\Leftarrow$:]
            
            考虑到 $A^\prime := \left(\begin{array}{cc}0&A\\A^\star&0\end{array}\right)$ 是Hermite矩阵,故其显然也是正规矩阵,酉相似于一对角阵,即存在 酉矩阵 $U_A$,
            $A^\prime = U_A \Lambda_A U_A^\star$. 同样地,对于Hermite阵 $B^\prime = \left(\begin{array}{cc}0&B\\B^\star&0\end{array}\right)$, 存在酉矩阵 $U_B$,
            $B^\prime = U_A \Lambda_B U_B^\star$. 注意到,有 $\mathtt{det}(xE - A^\prime) = \mathtt{det}(xE - \Lambda_A)$, $\mathtt{det}(xE - B^\prime) = \mathtt{det}(xE - \Lambda_B)$.
            
            由于 $A^\prime$ 酉相似于 $B^\prime$, 即存在酉矩阵 $U$, $UA^\prime U^\star = B^\prime$, 即 $UU_A\Lambda_AU_A^\star U^\star = U_B\Lambda_BU_B^\star$, 存在酉矩阵 
            $V = (U_B^\star UU_A)$ 使得 $V\Lambda_A V^\star = \Lambda_B$, 于是有 $\mathtt{det}(xE - \Lambda_A) = \mathtt{det}(xE - \Lambda_B)$.
            
            从而得到 $\mathtt{det}(xE - A^\prime) = \mathtt{det}(xE - B^\prime)$, 利用行变换将 $xE - A^\prime$ 与 $xE-B^\prime$消为下三角分块阵得: 

            $x^{n-m}\mathtt{det}(x^2E_m - A_{m\times n}A^\star_{n\times m}) = x^{n-m}\mathtt{det}(x^2E_m - B_{m\times n}B^\star_{n\times m})$.
            
            由此知 $\mathtt{det}(xE - AA^\star) = \mathtt{det}(xE - BB^\star)$,所以 $A$ 和 $B$ 有相同的正奇异值,即存在酉矩阵 $U^A_m, V^A_n, U^B_m, V^B_n$ 以及一个对角阵 $S_{m\times n}$, 使得 
            $A = U^A_mS_{m\times n}V^A_n$, $B = U^B_mS_{m\times n}V^B_n$, 得到 $A = (U^A_mU^{B\star}_m)B(V^{B\star}_nV^A_n)$, 这就证明了 $A$ 酉等价于 $B$.

        \end{itemize}

        \item [2.5 习题3]
        
        \begin{itemize}
            \item [(1)] 对于任一可逆矩阵 $A_n$, $A_n$ 列满秩,列空间为 $\mathbb{C}^n$,$A_n$亦有行满秩性质,故 $A^\star$ 列空间也为 $\mathbb{C}^n$. 故得可逆矩阵都是 EP-阵.
        
                        对于任一正规矩阵 $A_n$, 其酉相似于对角阵,即存在 酉矩阵 $U_n$ 使得 $UA = \mathtt{diag}(\lambda_1, \lambda_2,\cdots, \lambda_r, 0, \cdots, 0)U$, 其中 $r$ 为 $A$ 的列秩。
                        显然地,$UA^\star = \mathtt{diag}(\bar\lambda_1,\bar\lambda_2, \cdots, \bar\lambda_r, 0, \cdots, 0)U$, 故而 $A^\star$ 的列秩也是 $r$. 
                        
                        为证明它们对应的列向量空间相同,
                        记 $A = (\alpha_1, \cdots, \alpha_n)$, $A^\star = (\beta_1, \cdots, \beta_n)$ 存在可逆矩阵 $U_1$ 和 $U_2$ 使得 $UA = (U\alpha_1,\cdots,U\alpha_n) = \left(\begin{array}{cc}E_r&0\\0&0\end{array}\right)U_1$, 
                        $UA^\star = (U\beta_1,\cdots, U\beta_n) = \left(\begin{array}{cc}E_r&0\\0&0\end{array}\right)U_2$. 方便起见,记 $E_r = (\epsilon_1, \cdots, \epsilon_r)$. 
                        对$A$的列向量空间上的任一向量$v$, 有 
                        $$v = (\alpha_1, \cdots, \alpha_n) \gamma = U^{-1}(U\alpha_1, \cdots, U\alpha_n)\gamma = U^{-1} (U\beta_1, \cdots, U\beta_n)U_2^{-1}U_1\gamma = (\beta_1, \cdots, \beta_n)U_2^{-1}U_1\gamma$$, 其中 $\gamma \in\mathbb{C}^n$.
                        所以存在 $\gamma^\prime = U_2^{-1}U_1\gamma\in\mathbb{C}^n$, $v = (\beta_1, \cdots, \beta_n) \gamma^\prime$, 所以 $v$ 在 $A^\star$ 的列空间中,这就证明了 $A$ 的列空间是 $A^\star$ 的列空间的子空间。又知二者维数都为 $r$, 所以两列空间相同。(或者用反之亦然说明 $A^\star$ 列空间是 $A$ 列空间的子空间,进而说明相同)。
                        由此证得,正规矩阵都是 EP-阵.

            \item [(2)] 
            \begin{enumerate}
                \item [$\Leftarrow$:] 因为$B$是可逆矩阵,故可构造可逆矩阵 $Q_1 = \left(\begin{array}{cc}B&0\\0&E\end{array}\right)Q^\star$, $Q_2=\left(\begin{array}{cc}B^\star&0\\0&E\end{array}\right)Q^\star$, 
                
                有 $Q^\star A = \left(\begin{array}{cc}E_r&0\\0&0\end{array}\right)Q_1$, 
                                    $Q^\star A^\star = \left(\begin{array}{cc}E_r&0\\0&0\end{array}\right)Q_2$.
                由此,可完全仿照(1)中对正规矩阵的讨论,证明方阵 $A$ 是EP-阵。
                \item [$\Rightarrow$:] 对 $r$ 秩方阵$A$ 作 QR 分解, $A = Q_{n\times r}R_{r\times n}$, 其中 $Q$是列酉阵,$r(Q) = r(R) = r$, $Q^\star Q = E_r$.
                那么 $A^\star = R^\star Q^\star$. 由 $A$ 是 EP-阵知, $A^\star$ 的列向量可由 $Q$ 的列向量线性组合表示,故 $A^\star = QR^\prime$.
                由此得 $R^\star Q^\star = QR^\prime$, 故 $R^\prime = Q^\star QR^\prime = Q^\star R^\star Q^\star$, $A^\star = QR^\prime = QQ^\star R^\star Q^\star$, 于是 $A = QRQQ^\star$.
                关于$x\in\mathbb{C}^n$的方程 $Q^\star x = 0$ 其解空间是 $n-r$ 维线性空间,从中取出一组单位正交的基,作为列向量组成矩阵 $Q^\prime = (q_{r+1}, \cdots, q_n)$. 容易验证 $U = (Q; Q^\prime)$ 是 $n$ 阶的酉矩阵。
                于是有 $$A = (QR)(QQ^\star) = \left(\left(Q; Q^\prime\right)\times\left(\begin{array}{c}R\\0\end{array}\right)\right)\times\left(\left(Q;0\right)\times\left(\begin{array}{c}Q^\star\\Q^{\prime\star}\end{array}\right)\right) = U\left(\begin{array}{cc}RQ&0\\0&0\end{array}\right)U^\star$$
                且可以验证 $r(RQ) = r(R) = r(Q) = r$, $RQ$ 是 $r$ 阶可逆矩阵.
            \end{enumerate}
        \end{itemize}

        \item [2.5 习题5]
            \begin{enumerate}
                \item [(1)] 回顾奇异值分解存在性的构造证明可知, $V$ 的后$n-r$列是选取的 $A^\star Ax=0$ 的解空间的标准正交基。由于 $Ax=0\Rightarrow A^\star Ax=0$, 所以 $Ax=0$ 的解空间是 $A^\star Ax=0$ 解空间的子线性空间。
                又知二者均为 $n-r$ 维,故 $Ax=0$ 与 $A^\star Ax=0$ 二者解空间相同,故而 $V$ 的后 $n-r$ 列也是 $A$ 的解空间的一组标准正交基。 
                \item [(2)] $AV = U\Lambda$. 其中 $\Lambda = \left(\begin{array}{cc}\Lambda_r&0\\0&0\end{array}\right)$. 故可知 $U$ 的前 $r$ 列 是 $AV$ 的列空间的一组基,且由于是奇异值分解所以它是标准正交的。又由于 $V$ 是酉矩阵,故而 $AV$ 和 $A$ 的列空间相同,这就说明了 $U$ 的前 $r$ 列是 $A$ 的列空间的一个标准正交基。
                \item [(3)] 回顾奇异值分解存在性的构造证明可知,对于 $U=(\alpha_1,\cdots,\alpha_n)$ 的后 $n-r$ 列中的任意一列,比如 $\alpha_i, i>r$, 有 $AA^\star\alpha_i = \lambda_i\alpha_i = 0$ 因为 $\lambda_i = 0 , \forall i > r$. 仿(1) 可以得到 $AA^\star$ 和 $A^\star$ 的解空间相同,故而 $U$ 的后 $n-r$ 列是 $A^\star$ 解空间的子线性空间。
                            再由二者维数相同知,$U$ 的后 $n-r$ 列是该解空间的一组基,且由于$A=U\Lambda V^\star$是SVD分解,所以是标准正交的。
                \item [(4)] $A^\star = V\Lambda^\star U^\star$, 故 $A^\star U = V\Lambda^\star$, 同 (2) 可证。
            \end{enumerate}
        
        \item [2.6 习题1] 求 $A=\left(\begin{array}{ccccc}1&1&1&0&2\\1&0&1&1&3\\0&1&1&1&4\end{array}\right)$ 的MP广义逆。经计算得到:有下三角矩阵 $L = \left(\begin{array}{ccc}1&&\\8/7&1/7&\\10/7&9/70&1/10\end{array}\right)$ 以及行正交的矩阵 
        $Q = \left(\begin{array}{ccccc}1&1&1&0&2\\-1&-8&-1&7&5\\-13&6&-3&1&5\end{array}\right)$, 使得 $A=L Q$ . 可以验证 $QQ^\star = \mathtt{diag}(7, 140, 240)$. 下面验证 $B = Q^\star (QQ^\star)^{-1} L^{-1}$ 是 $A$ 的广义逆:
        \begin{enumerate}
            \item [a.] $ABA = LQ Q^\star (QQ^\star)^{-1} L^{-1} LQ = LQ = A$
            \item [b.] $BAB = Q^\star (QQ^\star)^{-1} L^{-1} LQ Q^\star (QQ^\star)^{-1} L^{-1} = Q^\star (QQ^\star)^{-1} L^{-1} = B$
            \item [c.] $AB = LQ Q^\star (QQ^\star)^{-1} L^{-1} = E$ 是 Hermite 矩阵
            \item [d.] $BA = Q^\star (QQ^\star)^{-1} L^{-1} LQ = Q^\star (QQ^\star)^{-1} Q$ 是 Hermite 矩阵,因为 $QQ^\star$ 是实对角阵
        \end{enumerate}
        带入具体数值即可求得 $A$ 的MP广义逆矩阵 
        $$B = Q^\star (QQ^\star)^{-1} L^{-1} = \left(\begin{array}{ccc}1&-1&-13\\1&-8&6\\1&-1&-3\\0&7&1\\2&5&5\end{array}\right)\times\left(
            \begin{array}{ccc}1/7&&\\&1/140&\\&&1/240\\\end{array}
        \right)\times\left(\begin{array}{ccc}1&&\\-8&7&\\-4&-9&10\end{array}\right)$$
        进一步化简得 $B = \dfrac{1}{48}\times\left(\begin{array}{ccc}20&21&-26\\24&-30&12\\12&3&-6\\-20&15&2\\-4&3&10\end{array}\right)$
    
        \item [习题2]
    
    $A = \left[\begin{array}{ccc}1&1&0\\0&1&5\end{array}\right]$, $AA^\top = \left[\begin{array}{cc}2&1\\1&26\end{array}\right]$,容易验证这是一个正定的 Hermite 矩阵,所以它的 Cholesky 分解存在. 证明了存在性后,Cholesky 分解的具体构造较为容易,这里直接给出结果:
    
    $L = \left[\begin{array}{cc}\sqrt{2} & 0 \\ \sqrt{1/2} & \sqrt{51/2}\end{array}\right]$, $AA^\top = LL^\top$.

    \item [习题3] 
    
    \begin{itemize}
        \item [(1)] 首先,容易验证 $AA^\star$ 是一个半正定的 Hermite 矩阵, 那么存在酉矩阵$U_m\in\mathbb{C}^{n\times n}$,使得
        
        $AA^\star = U_m \mathtt{diag}(\lambda_1, \dots, \lambda_r, 0, \dots, 0) U_m^\star$. 其中 $r = r(A) = r(A^\star) = r(AA^\star)$. 
        
        对 $A$ 进行奇异值分解,有 $A = U_m\left[\begin{array}{cc}\Lambda_r&0\\0&0\end{array}\right]_{m\times n} V_n$,其中 $V_n$ 是另一个 $n$ 阶酉矩阵, $\Lambda_r = \mathtt{diag}(\lambda_1^{1/2},\dots, \lambda_r^{1/2})$. 方便起见,记 $\Lambda = \mathtt{diag}(\lambda_1, \dots, \lambda_r, 0,\dots 0)$. 对SVD分解中的$m\times n$准对角矩阵进行分块可得: 
        
        $$A = U_m \left[ \Lambda^{1/2} ; O_{m\times (n - m)} \right] V_n = U_m \Lambda^{1/2}\left[E_m; O\right]V_n = U_m \Lambda^{1/2} U_m^\star U_m [E_m; O] V_n = PU_{m\times n} $$

        其中 $P = U_m \Lambda^{1/2} U_m^\star = (AA^\star)^{1/2}$, $U_{m\times n} = U_m [E_m; O] V_n = [V_m; O] V_n$, 可以验证 $U U^\star = [V_m; O] V_n V_n^\star [V_m; O]^\star = V_mV_m^\star = E_m$. 

        \item [(2)] $r(A) = m$ 时,$AA^\star$ 是满秩方阵, $P = (AA^\star)^{1/2}$ 也是满秩方阵,故而 $U = P^{-1}A$ 唯一确定。

    \end{itemize}

    \item [习题12] 证明关于方阵 $A\in\mathbb{M}_n$ 的下列三个命题的等价性:
    \begin{itemize}
        \item [(1)] 存在正整数 $k \geq 1$, 使得 $A^k = 0$ ;
        \item [(2)] 对于任意正整数 $m\geq 1$, $tr(A^m) = 0$ ;
        \item [(3)] 对于任意正整数 $m$, $1\leq m \leq n$, $tr(A^m) = 0$ .
    \end{itemize}
    \begin{proof}
        为方便讨论矩阵的迹, 根据 Schur 引理,对 $A$ 做分解: 存在酉矩阵 $U\in\mathbb{M}_n$, 和 上三角矩阵 $R\in\mathbb{M}_n$ 使得 $A = URU^\star$. 从而有 $A^k = UR^kU^\star, \forall k\in\mathbb{N}^*$. 通过简单的数学归纳可以证明 $R^k$ 是上三角矩阵,且 $(R^k)_{ii} = (R_{ii})^k$. 下面按照 (1) $\Rightarrow $ (2) $\Rightarrow $ (3) $\Rightarrow$ (1) 的顺序来证明等价性:
        \begin{itemize}
            \item [(1) $\Rightarrow$ (2)] $A^k = UR^kU^\star = 0$, 其中酉矩阵 $U$ 可逆, 所以 $R^k = 0$. 于是对于任意的 $1\leq i\leq n, i\in\mathbb{N}$, $(R^k)_{ii} = (R_{ii})^k = 0$, 所以 $R_{ii} = 0$ 即 $R$ 的对角线元素均为 $0$. 所以 对正整数 $m\geq 1$, $tr(R^m) = \sum_i R_{ii}^m = 0$, $tr(A^m) = tr(UR^mU^\star) = tr(R^mU^\star U) = tr(R^m) = 0$.
            \item [(2) $\Rightarrow (3)$] 显然
            \item [(3) $\Rightarrow (1)$] 对于任意的 $1\leq m\leq n$, $tr(A^m) = tr(R^m) = \sum_{i} R_{ii}^m = 0$. 假设 $R_{ii}, 1\leq i\leq n$ 不全为 $0$, 则可取出其中不为 $0$ 的项,去重后得到 $r_1, r_2, \cdots, r_t$, $1\leq t \leq n$. 从而有方程组
            $$\left[\begin{array}{cccc}
                r_1 & r_2 & \cdots & r_t\\
                r_1^2 & r_2^2 & \cdots & r_t^2\\
                \vdots & \vdots & \ddots & \vdots\\
                r_1^{t} & r_2^{t} & \cdots & r_t^{t} 
            \end{array}\right]
            \left[\begin{array}{c}
                n_1 \\ n_2 \\ \vdots \\ n_t
            \end{array}\right] = 
            \left[\begin{array}{cccc}
                1 & 1 & \cdots & 1\\
                r_1^1 & r_2^1 & \cdots & r_t^1\\
                \vdots & \vdots & \ddots & \vdots\\
                r_1^{t-1} & r_2^{t-1} & \cdots & r_t^{t-1} 
            \end{array}\right]
            \left[\begin{array}{cccc}
                r_1 & & & \\
                & r_2 & & \\
                & & \ddots & \\
                & & & r_t
            \end{array}\right]
            \left[\begin{array}{c}
                n_1 \\ n_2 \\ \vdots \\ n_t
            \end{array}\right] = 
            \left[\begin{array}{c}
                0 \\ 0 \\ \vdots \\ 0
            \end{array}\right]$$
            其中 $n_i$ 表示 $r_i$ 去重前的出现次数,应有 $r_i\ne r_j, \forall i\ne j$ 以及 $n_i > 0, r_i \ne 0, \forall i$.
            故而此时 方程中的 Vandermonde 矩阵和对角矩阵均可逆,从而$[n_1, n_2, \cdots, n_t]^\top$应为 $0$ 向量,矛盾。所以 $R_{ii}$ 不全为 $0$ 的假设不成立,从而得到 $R$ 是对角线全为 $0$ 的上三角矩阵。即 $R_{ij} = 0, \forall i < j + 1$. 
            
            下归纳证明 $(R^{k})_{ij} = 0, \forall i < j + k$ : (1) 对于 $k=1$ 成立;(2) $(R^k)_{ij} = 0, \forall i < j + k$ $\Rightarrow$ $(R^{k+1})_{ij} = (R^k R)_{ij} = \sum_t (R^k)_{it} R_{tj}$ . 当 $i < j + k + 1$ 时, $i \geq t + k$ 与 $t \geq j + 1$ 不能同时成立,故 $(R^k)_{it}$ 与 $R_{tj}$ 中至少有一个为 $0$, 从而推出 $(R^{k+1})_{ij} = 0, \forall i < j + (k + 1)$.

            对于任意 $1\leq i,j\leq n$, 有 $i < j + n$, 于是 $(R^n)_{ij} = 0$, 所以 $R^n = 0$, $A^n = UR^n U^\star = 0$.

            注:这还说明了,如果存在正整数 $k$ 使得 $A^k = 0$, 那么存在 $k \leq n$ 使得 $A^k = 0$.
        \end{itemize}
    \end{proof}

    \item [1.] 习题 1
    
        $\sigma: A \to AB^\top + BA$, 注意到 $BA = (A^\top B^\top)^\top = (AB^\top)^\top$, 所以  $AB^\top + BA$ 是$2$阶的实对称矩阵,仍在 $V$ 中。 

        \begin{itemize}
            \item [(1)] 对任意的 $\lambda_1, \lambda_2 \in\mathbb{R}$, $A_1, A_2\in V$, 有 $\sigma(\lambda_1 A_1 + \lambda_2 A_2) = (\lambda_1 A_1 + \lambda_2 A_2) B^\top + B(\lambda_1 A_1 + \lambda_2 A_2)= \lambda_1\cdot(A_1B^\top + BA_1) + \lambda_2\cdot (A_2B^\top + BA_2) = \lambda_1 \sigma(A_1) + \lambda_2 \sigma(A_2)$.
            \item [(2)] 由于 $\sigma(E_{11}) = 2E_{11}$, $\sigma(E_{12} + E_{21}) = -2 E_{11} + (E_{12} + E_{21})$, $\sigma(E_{22}) = - (E_{12} + E_{21})$. 故 $\sigma$ 在这组基下,对应于矩阵 $\left[\begin{array}{ccc}2&-2&0\\0&1&-1\\0&0&0\end{array}\right]$
            \item [(3)] 对于任意 $A\in V$, $A = [E_{11}, E_{12} + E_{21}, E_{22}]\cdot [c_1, c_2, c_3]^\top, c_i\in\mathbb{R}$,
            
            $\sigma A = \sigma[E_{11}, E_{12} + E_{21}, E_{22}] \cdot [c_1, c_2, c_3]^\top = [E_{11}, E_{12} + E_{21}, E_{22}] \cdot \left[\begin{array}{ccc}2&-2&0\\0&1&-1\\0&0&0\end{array}\right]\cdot [c_1, c_2, c_3]^\top = [E_{11}, E_{12} + E_{21}, E_{22}]\cdot \left[2c_1 - 2c_2, c_2 - c_3, 0\right]^\top$, 由此可知 $\{E_{11}, E_{12} + E_{21}\}$ 是像子空间 $\mathtt{im}(\sigma)$ 的一组基。
            
            \item [(4)] 延用 (3) 中的记号,由于 $E_{11}, E_{12} + E_{21}, E_{22}$ 线性无关,所以 $\sigma A = 0$ 当且仅当 $\left[\begin{array}{ccc}2&-2&0\\0&1&-1\\0&0&0\end{array}\right] \left[\begin{array}{c}c_1\\c_2\\c_3\end{array}\right] = 0$, 解为 $[c_1, c_2, c_3]^\top = [c_3, c_3, c_3]^\top, c_3\in\mathbb{R}$. 故核子空间 $\mathtt{ker}(\sigma)$ 的一组基是 $\{E_{11} + E_{12} + E_{21} + E_{22}\}$.
            \item [(5)] 是直和,且 $V = \mathtt{im}(\sigma) \oplus \mathtt{ker}(\sigma)$. 因为 $\mathtt{im} (\sigma)$ 与 $\mathtt{ker}(\sigma)$ 的基不交,且它们基的并是 $V$ 的一组基。
        \end{itemize}

    \item [2.] 习题 3
    
    设 $\{\eta_1, \eta_2, \cdots, \eta_n\}$ 是 $n$ 维空间 $\mathbb{F}^{n}$ 的一组基。对于线性变换 $\sigma$, 以及任意的 $\alpha\in \mathbb{F}^n$, $\alpha = [\eta_1, \eta_2, \cdots, \eta_n]\cdot c^\top$, 其中 $c = [c_1, c_2, \cdots, c_n] \in\mathbb{F}^n$. $\sigma(\alpha) = \sigma [\eta_1, \eta_2, \cdots, \eta_n]\cdot c^\top$. 由于 $\{\eta_i\}$ 是基,故存在矩阵 $A$ 使得 $\sigma [\eta_1, \cdots, \eta_n] = [\eta_1, \cdots, \eta_n] \cdot A$. 再记 $N = [\eta_1, \eta_2, \cdots, \eta_n]$, 显然 $N$ 可逆, 构造 $B = NAN^{-1}\in M_n(\mathbb{F})$, 有 $B\alpha = B(Nc^\top) = NAN^{-1}Nc^\top = NAc^\top = \sigma(\alpha)$.

    \item [3.] 习题 4
    
    \begin{itemize}
        \item [(1)] $V = W_1 \oplus W_2$, $\sigma: \alpha_1 + \alpha_2 \to \alpha_1$, 其中 $\alpha_i\in W_i$. 取 $W_1$ 的一组基 $(\alpha_1, \cdots, \alpha_r)$; 取 $W_2$ 的一组基 $(\beta_1, \cdots, \beta_s)$. 那么 $(\alpha_1, \cdots, \alpha_r, \beta_1, \cdots, \beta_s)$ 是 $V$ 的一组基。对于 $V$ 的任意基 $(\gamma,\cdots, \gamma_r, \gamma_{r+1},\cdots, \gamma_{r+s})$, 有 $(\gamma,\cdots, \gamma_r, \gamma_{r+1},\cdots, \gamma_{r+s}) = (\alpha_1, \cdots, \alpha_r, \beta_1, \cdots, \beta_s) C$, 其中 $C\in M_{r+s}(\mathbb{F})$. 将 $\sigma$ 作用于其上:$$\begin{array}{ll}\sigma (\gamma,\cdots, \gamma_r, \gamma_{r+1},\cdots, \gamma_{r+s}) &= \sigma (\alpha_1, \cdots, \alpha_r, \beta_1, \cdots, \beta_s)\cdot C \\ &= (\alpha_1, \cdots, \alpha_r, \beta_1, \cdots, \beta_s) \left[\begin{array}{cc}E_r&0\\0&0\end{array}\right]C\\&=(\gamma,\cdots, \gamma_r, \gamma_{r+1},\cdots, \gamma_{r+s})\cdot C^{-1} \left[\begin{array}{cc}E_r&0\\0&0\end{array}\right]C\end{array}$$
        从而得到 $A = C^{-1} \left[\begin{array}{cc}E_r&0\\0&0\end{array}\right] C$, 显然有 $A^2 = A$.

        \item [(2)] 
        \begin{itemize}
            \item [(a)] 即 $\sigma_A: (\alpha_1, \cdots, \alpha_n) \beta \to (\alpha_1,\cdots,\alpha_n) (A\beta)$. 容易验证 对任意 $\lambda_1, \lambda_2 \in \mathbb{F}$, 以及 任意 $v_1, v_2\in V$, 有 $\exists \beta_1, v_1 = (\alpha_1, \cdots, \alpha_n) \beta_1$, $\exists \beta_2, v_2 = (\alpha_1, \cdots, \alpha_n) \beta_2$, $\sigma(\lambda_1 v_1+ \lambda_2 v_2) = (\alpha_1, \cdots, \alpha_n) (A\lambda_1\beta_1 + A\lambda_2\beta_2) = \lambda_1(\alpha_1, \cdots, \alpha_n)(A\beta_1) + \lambda_2(\alpha_1, \cdots, \alpha_n)(A\beta_2) = \lambda_1 \sigma(v_1) + \lambda_2 \sigma(v_2)$, 所以 $\sigma$ 是线性变换
            \item [(b)] $A^2 = A \Rightarrow A(A-E) = (A-E)A = 0$. 下考察 $r(A-E)$ 与 $r(A)$ 的关系. 设 $r(A) = r$, 则存在可逆矩阵 $P, Q\in M_n(\mathbb{F})$ 使得 $A = P\left[\begin{array}{cc}E_r&0\\0&0\end{array}\right]Q$, 由于 $A^2 = P\left[\begin{array}{cc}E_r&0\\0&0\end{array}\right]QP\left[\begin{array}{cc}E_r&0\\0&0\end{array}\right]Q=A$, 所以$QP$具有 $QP=\left[\begin{array}{cc}E_r&C_{12}\\C_{21}&C_{22}\end{array}\right]$的形式。$A - E = P\left[\begin{array}{cc}E_r&0\\0&0\end{array}\right]Q - PP^{-1}Q^{-1}Q = P\left(\left[\begin{array}{cc}E_r&0\\0&0\end{array}\right] - \left[\begin{array}{cc}E_{r}&C_{12}\\C_{21}&C_{22}\end{array}\right]^{-1}\right)Q$. 注意到 $\left[\begin{array}{cc}E_r&C_{12}\\C_{21}&C_{22}\end{array}\right]^{-1} = \left[\begin{array}{cc}E_r + C_{12}D^{-1}C_{21}&-C_{12}D^{-1}\\-D^{-1}C_{21}&D^{-1}\end{array}\right]$ , 其中 $D = C_{22} - C_{21}C_{12}$ 是($n-r$)阶可逆矩阵 (可逆性由 $QP$ 可逆推出). 所以: 
            $\left[\begin{array}{cc}E_r&0\\0&0\end{array}\right] - \left[\begin{array}{cc}E_r&C_{12}\\C_{21}&C_{22}\end{array}\right]^{-1}=\left[\begin{array}{c}-C_{12}\\E_{n-r}\end{array}\right] D^{-1} \left[C_{21}; -E_{n-r}\right]$, 从而 $r(A-E) = n-r$.

            下面验证 $A$ 的列向量空间和 $A-E$ 的解空间相同,即 $\mathtt{col}(A) = (A-E)^\perp$. $v\in \mathtt{col}(A)\Rightarrow \exists \gamma, v = A\gamma \Rightarrow (A-E)v = (A-E)A\gamma = 0 \Rightarrow v\in (A-E)^\perp$, 故 $\mathtt{col}(A)\subset (A-E)^\perp$. 又 $\dim(\mathtt{col}(A)) = r(A) = r = n - r(A-E) = \dim\left((A-E)^\perp\right)$, 所以 $\mathtt{col}(A) = (A-E)^\perp$. 同理可得,$\mathtt{col}(A-E) = A^\perp$.

            令 $W_1 = \{(\alpha_1,\cdots,\alpha_n)\beta | \forall \beta\in A^{\perp}\}$, $W_2 = \{(\alpha_1, \cdots, \alpha_n)\beta | \forall \beta\in (A-E)^\perp\}$ (或等价地,$W_1 = \{(\alpha_1,\cdots,\alpha_n)\beta | \forall \beta\in \mathtt{col}(A-E)\}$, $W_2 = \{(\alpha_1, \cdots, \alpha_n)\beta | \forall \beta\in \mathtt{col}(A)\}$), 则 $V = W_1 \oplus W_2$ 是满足题意的一个直和分解,验证如下:
            \begin{itemize}
                \item [(i)] 设 $v = (\alpha_1,\cdots,\alpha_n)\beta\in W_1\cap W_2$ 则 $\beta \in A^\perp \cap (A-E)^\perp$, 于是 $\beta = A\beta = 0$, $x = 0$. 所以 $W_1 \cap W_2 = 0$.
                \item [(ii)] $\dim(W_1) + \dim(W_2) = \dim(A^\perp) + \dim((A-E)^\perp) = (n - r) + r = n = \dim (V)$, 结合 (i) 知,$V = W_1 \oplus W_2$.
                \item [(iii)] $\forall \alpha = (\alpha_1,\cdots,\alpha_n)\beta\in W_1$, 有 $\sigma \alpha = (\alpha_1,\cdots,\alpha_n) (A\beta) = 0$ 因为 $\beta\in A^\perp, A\beta=0$. 类似地,$\forall \alpha = (\alpha_1,\cdots,\alpha_n)\beta \in W_2$, 有 $\sigma \alpha = (\alpha_1, \cdots, \alpha_n) (A\beta) = \alpha$ 因为 $\beta\in (A-E)^\perp, A\beta = \beta$. 
            \end{itemize}
            (好像$W_1$, $W_2$和题目中的顺序反了,算了不改了)
        \end{itemize}
    \end{itemize}

    \item [4.] 习题 5
    
    设线性变换 在 基 $\{\alpha_1, \alpha_2, \cdots, \alpha_n\}$ 下的对应的矩阵 为 $A$, 在基 $\{\beta_1, \beta_2, \cdots, \beta_n\}$ 下对应的矩阵为 $B$. 根据定义有:$\sigma (\alpha_1, \cdots, \alpha_n) = (\alpha_1, \cdots, \alpha_n) A$, $\sigma (\beta_1, \cdots, \beta_n) =(\beta_1, \cdots, \beta_n) B$. 考虑到同一线性空间的基之间能互相表示,故存在可逆矩阵 $Q$ 使得 $(\beta_1, \cdots, \beta_n) = (\alpha_1, \cdots, \alpha_n)Q$, 从而有 $$\begin{array}{ll}\sigma(\beta_1, \cdots, \beta_n) &= (\beta_1, \cdots, \beta_n)B = (\alpha_1,\cdots,\alpha_n)QB\\ &= \sigma [(\alpha_1, \cdots, \alpha_n) Q] = (\sigma\alpha_1, \cdots, \sigma\alpha_n)Q = (\alpha_1,\cdots,\alpha_n) AQ\end{array}$$
    故而 $(\alpha_1, \cdots, \alpha_n) (QB - AQ) = 0$, 由于基线性无关,故 $QB - AQ = 0 \Rightarrow A = QBQ^{-1}$, 即 $A$ 与 $B$ 相似。

    \item [1.] 习题2
    
    \begin{itemize}
        \item [$\Rightarrow$:] 不妨设 $\sigma: V = W \oplus W^\perp \to W$, 其中 $(w_1, \cdots, w_r)$ 是 $W$ 子线性空间的一组标准正交基, $(w_{r+1}, \cdots, w_{n})$ 是 $W^\perp$ 上的一组标准正交基,显然 $(w_1, \cdots, w_{r}, w_{r+1}, \cdots w_n)$ 构成 $V = W\oplus W^\perp$ 的一组标准正交基。对于 $V$ 上的任意一组标准正交基 $\alpha_1, \cdots, \alpha_n$, 存在可逆矩阵 $Q$ 使得 $(\alpha_1, \cdots, \alpha_n) = (w_1, \cdots, w_n) Q$, 由于两组都是标准正交基,故而 $Q$ 是酉矩阵。考虑 设 $\sigma$ 在标准正交基 $\{\alpha_i\}$ 下对应矩阵 $A$, 则有
        
        $\begin{array}{ll}(\alpha_1, \cdots, \alpha_n) A=\sigma (\alpha_1, \cdots, \alpha_n) &= \sigma(w_1, \cdots, w_r, w_{r+1}, \cdots, w_n) Q \\&= (w_1, \cdots, w_r, w_{r+1}, \cdots, w_n) \left[\begin{array}{cc}E_r&0\\0&0\end{array}\right]Q\\&=(\alpha_1, \cdots, \alpha_n)Q^\star \left[\begin{array}{cc}E_r&0\\0&0\end{array}\right]Q\end{array}$

        所以 $A = Q^\star \left[\begin{array}{cc}E_r&0\\0&0\end{array}\right]Q$ 是 Hermite 矩阵。又 $A^2 = Q^\star \left[\begin{array}{cc}E_r&0\\0&0\end{array}\right]QQ^\star \left[\begin{array}{cc}E_r&0\\0&0\end{array}\right]Q = Q^\star \left[\begin{array}{cc}E_r&0\\0&0\end{array}\right]Q = A$, 所以$A$是幂等的。
        \item [$\Leftarrow$:] 不妨设 $\sigma$ 在某组标准正交基 $\alpha_1, \cdots, \alpha_n$ 下对应矩阵 $A$, $A$ 是幂等的Hermite矩阵。对于 Hermite 矩阵 $A$, 存在 酉矩阵 $U = (u_1, \cdots, u_n)$ 以及实对角矩阵 $\Lambda$, 使得 $A = U\Lambda U^\star$. 由于是幂等的,$A^2=U\Lambda^2 U^\star = A =U\Lambda U^\star$, 即 $\Lambda^2 = \Lambda$. 所以 $\Lambda$ 对角线元素为 $1$ 或 $0$. 不妨设 $\Lambda = \left[\begin{array}{cc}E_r&0\\0&0\end{array}\right]$, 则 $A = \sum_{i=1}^r u_i u_i^\star$. 容易验证 $\forall i \leq r, Au_i = u_i$, $\forall i > r, Au_i = 0$.
        
        构造 $W = \{(\alpha_1, \cdots, \alpha_n) x | x = \sum_{i=1}^r c_i u_i, c_i\in\mathbb{F}\}$, 从而 $W^\perp = \{(\alpha_1, \cdots, \alpha_n) x | x = \sum_{i=r+1}^n c_i u_i, c_i\in\mathbb{F}\}$, 容易验证 $V = W \oplus W^\perp$, 且 $\sigma$ 是从 $V$ 到 $W$ 的一个正交投影变换, 即 $\forall v\in W, \sigma v = v$, $\forall v\in W^\perp, \sigma v = 0$.
    \end{itemize}

    \item [2.] 习题5
    
    \begin{itemize}
        \item [(1)] $[A, B]:= tr(A^\star B) = \sum_{i} (A^\star B)_{ii} = \sum_{i}\sum_{k} (A^\star)_{ik}B_{ik} = \sum_i\sum_k \overline{A_{ki}}B_{ki}$. 容易验证: 
        \begin{itemize}
            \item [*] 对称性:$[B, A] = \sum_{i}\sum_{k} \overline{B_{ki}}A_{ki} = \sum_i\sum_k \overline{\overline{A_{ki}}B_{ki}} = \overline {[A, B]}$.
            \item [*] 线性性:$[A, \alpha C+ \beta D] = \sum_i\sum_k \overline{A_{ki}}(\alpha C_{ki} + \beta D_{ki}) = \alpha \sum_i\sum_k \overline{A_{ki}}C_{ki} + \beta\sum_i\sum_k \overline{A_{ki}}D_{ki} = \alpha [A, C] + \beta [A, D]$.
            \item [*] 正定性:$[A, A] = \sum_i\sum_k \overline{A_{ki}}A_{ki} = \sum_i\sum_k |A_{ki}|_2^2 \geq 0$, 当且仅当 $A = 0$ 时取等号。
        \end{itemize}
        \item [(2)] $E_{11} = \left[\begin{array}{cccc}1&0&0&0\\0&0&0&0\\0&0&0&0\\0&0&0&0\end{array}\right]$, $ee^\top = \left[\begin{array}{cccc}1&1&1&1\\1&1&1&1\\1&1&1&1\\1&1&1&1\end{array}\right]$. 记夹角为 $\theta$, 有 $\cos \theta = [E_{11}, ee^\top] / \sqrt{[E_{11}, E_{11}][ee^\top, ee^\top]} = 1 / \sqrt{1\times 16} = 1/4$, 即 $\theta = \arccos (1/4)$.
        
        该内积空间的一个标准正交基 $\{E_{ij} | i, j\in\{1,2,3,4\}\}$.
    \end{itemize}

    \item[3.] 习题6
    
    \begin{itemize}
        \item [(1)] 容易验证 $E_{11}, E_{12} + E_{21}, E_{22}$ 是 $W$ 的一组基。所以 $B \in W^\perp$, 当且仅当 $[E_{11}, B] = 0, [E_{12} + E_{21}, B] = 0, [E_{22}, B] = 0$. 设 $B = b_{11}E_{11} + b_{12}E_{12} + b_{21}E_{21} + b_{22}E_{22}$, 则有 
        
        $\left[\begin{array}{cccc}1&0&0&0\\0&1&1&0\\0&0&0&1\end{array}\right]\left[\begin{array}{c}b_{11}\\b_{12}\\b_{21}\\b_{22}\end{array}\right] = \left[\begin{array}{c}0\\0\\0\\0\end{array}\right]$, 所以 $[b_{11}, b_{12}, b_{21}, b_{22}]^\top = [0, t, -t, 0]^\top, t\in\mathbb{R}$.

        所以正交补子空间 $W^\perp = \{t(E_{12} - E_{21}) | t\in\mathbb{R}\}$.

        \item [(2)] 由于 $\left[\begin{array}{cc}1&1\\0&0\end{array}\right] = A + B = \left[\begin{array}{cc}1&0.5\\0.5&0\end{array}\right] + \left[\begin{array}{cc}0&0.5\\-0.5&0\end{array}\right]$, 其中 $A\in W, B\in W^\perp$, 所以在$W$上的正交投影为 $A$, 即 $\left[\begin{array}{cc}1&0.5\\0.5&0\end{array}\right]$.
        
    \end{itemize}

    \item [4.] 补充 例4.13
    
    对于 $\forall \beta\in \mathtt{im}(g)$, 存在 $\gamma\in V$, $g\gamma = v$. 所以 $f\beta = f(g\gamma) = (fg) \gamma = 0$, 即 $\beta \in \mathtt{ker}(f)$,
    故 $\mathtt{im}(g)\subseteq \mathtt{ker}(f)$.

    根据课上讲到的结论 (或者考查 $g$ 在某组基下对应的矩阵 $A$ 并运用作业7中证明的 55页 习题4 的结论) 有:
    $r(g) + r(I_V - g) = n$. 容易验证 $\mathtt{im}(I_V - g) \subseteq \mathtt{ker}(g)$, 因为对任意 $(I_V - g)\gamma, \gamma\in V$ 有 $g(I_V-g)\gamma = 0$. 而 $\dim\mathtt{ker}(g) = n-r(g) = r(I_V-g)=\dim\mathtt{im}(I_V-g)$, 所以有 $\mathtt{ker}(g) = \mathtt{im}(I_V - g)$. 
    
    考虑任意 $x\in \mathtt{im}(g) \cap\mathtt{im}(I_V - g) = \mathtt{im}(g) \cap\mathtt{ker}(g)$, 有 $x = g\gamma$, 且 $0 = gx = g^2\gamma = g\gamma = x$. 所以 $\mathtt{im}(g)\cap\mathtt{im}(I_V - g) = \{0\}$. 所以有 $V = \mathtt{im}(g) \oplus \mathtt{im}(I_V-g)$.
    \begin{itemize}
        \item [(1)] 对于任意 $v\in V - \mathtt{ker}(f)$, 由于 $v\notin \mathtt{ker}(f)$, $\mathtt{im}(g)\subseteq \mathtt{ker}(f)$, 所以 $v\notin \mathtt{im}(g)$. 又 $V = \mathtt{im}(g) \oplus \mathtt{im}(I_V - g)$, 所以 $v\in \mathtt{im}(I_V - g) = \mathtt{ker}(g)$. 这就验证了 $V = \mathtt{ker}(f) + \mathtt{ker}(g)$.
        
        \item [(2)] 
        \begin{itemize}
            \item [$\Rightarrow$:] 若 $V = \mathtt{ker}(f) \oplus \mathtt{ker}(g)$, 则 $n = \dim\mathtt{ker}(f) + \dim \mathtt{ker}(g) = n-r(f) + n - r(g)$, 故 $r(f) + r(g) = n$.
            \item [$\Leftarrow$:] 若 $r(f) + r(g) = n$, 则 $\dim\mathtt{ker}(f) = n - r(f) = r(g) = \dim\mathtt{im}(g)$. 又 $\mathtt{im}(g)\subseteq \mathtt{ker}(f)$, 故此时 $\mathtt{ker}(f) = \mathtt{im}(g)$. 又因为 $\mathtt{ker}(g) = \mathtt{im}(I_V - g)$, 所以 $V = \mathtt{im}(g) \oplus \mathtt{im}(I_V - g) = \mathtt{ker}(f) \oplus \mathtt{ker}(g)$.
        \end{itemize}
        
    \end{itemize}

    \item [1.] 59页 习题4 
    
    \begin{lemma}
        正交矩阵可以由若干镜面反射矩阵相成得到
    \end{lemma}
    \begin{proof}
        设 $Q\in M_n(\mathbb{R})$, $Q=(q_1, q_2, \cdots, q_n)$. 由于 $(q_1, q_1) = 1$, 故存在镜面反射矩阵 $U_1$ 使得 $U_1 q_1 = e_1 := (1, 0, \cdots, 0)^\top$. 对于任意 $j\ne 1$, $(U_1q_j, U_1q_1) = q_j^\top U_1U_1 q_1 = q_j^\top q_1 = 0$, 所以 $U_1q_j = (0, *, \cdots, *)^\top$, 即第一个分量必然为 $0$. 从而有 $$U_1Q = (U_1q_1, U_1q_2, \cdots, U_1q_n) = \left[\begin{array}{cc}1&0\\0&Q^\prime\end{array}\right]$$

        由于镜面反射变换不改变内积,即 $(U_1q_i, U_1q_j) = (q_i, q_j)$, 故 $Q^\prime$ 是 $n-1$ 阶的正交矩阵。归纳地进行下去即可得到 $U_{n-1}\cdots U_2U_1Q = E_n$, 即 $Q = U_1U_2\cdots U_{n-1}$.
    \end{proof}

    对于 正交变换 $\sigma$ 它在 $V$ 的一组标准正交基 $\eta_1, \cdots, \eta_n$ 下对应于 矩阵 $Q$, 容易验证 $Q$ 是正交矩阵。

    那么 $\sigma(\eta_1, \cdots, \eta_n) = (\eta_1, \cdots, \eta_n) Q = (\eta_1, \cdots, \eta_n)U_1U_2\cdots U_{n-1} $, 其中 $U_i$ 是镜面反射矩阵

    在标准正交基 $\eta_1, \cdots, \eta_n$ 下,镜面反射矩阵 $U_1$ 对应于镜面反射变换 $\sigma_1$, 于是 $(\eta_1, \cdots, \eta_n)U_1U_2\cdots U_{n-1} = \sigma_1(\eta_1, \cdots, \eta_n) U_2\cdots U_{n-1}$. 由于镜面反射变换不改变内积,故而 $\sigma_1(\eta_1,\cdots,\eta_n)$ 仍是一组标准正交基,不妨设 $U_2$ 在 $\sigma_1\{\eta_i\}$ 下对应于 镜面反射变换 $\sigma_2$, 则有 $(\eta_1, \cdots,\eta_n)U_1U_2\cdots U_{n-1} = (\sigma_2\sigma_1)(\eta_1,\cdots,\eta_n)U_3\cdots U_{n-1}$. 归纳地进行下去,假设镜面反射矩阵 $U_{i+1}$ 在 $(\sigma_i\cdots \sigma_1)(\eta_1, \cdots, \eta_n)$ 这一标准正交基 下对应于 镜面反射变换 $\sigma_{i+1}$, 则最终得到 
    $$\sigma(\eta_1,\cdots,\eta_n) = (\sigma_{n-1}\sigma_{n-2}\cdots\sigma_1)(\eta_1,\cdots,\eta_n)$$

    这就构造出了 $\sigma = \sigma_{n-1}\sigma_{n-2}\cdots \sigma_1$.

    似乎证复杂了,实际上只需要假设 $U_i$ 在 $(\eta_1, \cdots, \eta_n)$ 下对应于镜面反射变换 $\sigma_i$, 即可得到 $\sigma = \sigma_1\sigma_2\cdots \sigma_{n-1}$

    \item [2.] 60页 习题8

    方便起见,这里先证明第2问的结论,以说明 $\tau$ 的存在性,再回到第1问,补充证明其唯一性:

    $\sigma$ 在 基 $V=(v_1,v_2, \cdots, v_n)$ 下对应于矩阵 $A$, 取 $\tau$ 为 在这组基下 $A^\star$ 对应的线性变换,任取 $v = (v_1, \cdots, v_n)\alpha$, $w = (v_1,\cdots, v_n)\beta$, 则 $\sigma v = (v_1,\cdots, v_n)A\alpha$, $\tau w = (v_1,\cdots, v_n)A^\star\beta$. 
    
    计算得到 $[\sigma v, w] = \alpha^\star A^\star V^\star V \beta$, $[v, \tau w] = \alpha^\star V^\star VA^\star \beta$.
    
    实际上应该需要增加 $(v_1, \cdots, v_n)$ 是标准正交基的条件,以保证 $V^\star V = E$, 从而有 $[\sigma v, w] = [v, \tau w]$.

    下面验证这样的 $\tau$ 的唯一性:假设 $\tau_1, \tau_2$ 都满足 $[\sigma v, w] = [v, \tau_i w], \forall v, w\in V, i=1,2$. 
    
    构造线性变换 $\tau^\prime : x \to \tau_1 x - \tau_2 x$, 则有 $\forall  v, w\in V, [v, \tau^\prime w] = 0$. 取 $v=\tau^\prime w$ 即有 对任意的 $w\in V$, $[\tau^\prime w, \tau^\prime w] = 0$, $\tau^\prime w = 0$, 这就验证了 $\tau^\prime = \tau_1 - \tau_2$ 是零线性变换, 故 $\tau_1 = \tau_2$.

    若 $\sigma\sigma^\star = \sigma^\star\sigma$ 则称 $\sigma$ 是正规线性变换,这一定义当然与正规矩阵的概念和谐。因为正规矩阵是指使得 $AA^\star = A^\star A$ 成立的矩阵,而在标准正交基下,线性变换与矩阵是对应的。

    \item [3.] 63页 习题3 
    \begin{itemize}
        \item [$\Rightarrow$:] 由教材 62页 命题 3.4.3 直接可得;
        \item [$\Leftarrow$:]
        \begin{lemma}
            如果 $V = A\oplus W_A = B \oplus W_B$, 且有 $B\subseteq W_A$, 那么 $W_A = B\oplus (W_A \cap W_B)$.
        \end{lemma}
        \begin{proof}
            显然有 $W_A\cap W_B \subseteq W_B$ 与 $B$ 的交集为 $\{0\}$, 所以 $B + (W_A\cap W_B) = B\oplus (W_A \cap W_B)$.
            
            
            考虑到 $B\subseteq W_A, (W_A\cap W_B)\subseteq W_A$, 所以 $B\oplus(W_A\cap W_B) \subseteq W_A$. 下验证 $W_A\subseteq B\oplus(W_A\cap W_B)$:
            
            对于任意 $v\in W_A\subseteq V = B \oplus W_B$, $v = b + w_b$, 其中 $b\in B, w_b\in W_B$. 假设 $w_b$ 在 $V=A \oplus W_A$ 下表示为 $w_b = a + w_a$, $a\in A, w_a\in W_A$, 那么有 $v = b + a + w_a$. 由于 $b\in B\subseteq W_A, w_a\in W_A$, $W_A$ 是线性空间, 所以 $a = v - b - w_a \in W_A$, 又 $a\in A$, 所以 $a=0$. 从而 $w_b = w_a\in W_A$, 即 $w_b\in (W_A\cap W_B)$. 这就验证了 $W_A$ 中任意的 $v$ 可以表示为 $v = b + w_b$, 其中 $b\in B, w_b\in(W_A\cap W_B)$.
        \end{proof}
        假设 $\lambda$ 是 $\sigma$ 在 $V$ 下的某一个特征值,设 $A=\{v\in V| \sigma v = \lambda v\}$. 
        
        如果 $\dim A = \dim V$, 那么$\sigma$ 在某组基下对应于对角阵 $\lambda E$. 
        
        如果 $0 < \dim A < \dim V$, 显然 $A$ 是一个 $\sigma$-子空间,根据题设,它存在 $\sigma$-子空间直和补 $W_A$, $V = A\oplus W_A$. 由 Lemma 0.2 可知 $W_A$ 也满足 “每个 $\sigma$-子空间 都有一个 $\sigma$-子空间直和补”,因为对于 $W_A$ 的 $\sigma$-子空间 $B$, 在 $V$ 上存在 $\sigma$-子空间直和补 $W_B$, 从而可以构造 $W_A$ 上的 $\sigma$-子空间直和补 $W_A\cap W_B$. 由于 $\dim W_A < \dim V$, 归纳地进行下去,即可证得 $\sigma$ 在 $W_A$ 的某个基下对应于对角阵 $\Lambda_{\dim W_A}$, 从而 证得 $\sigma$ 在 $V$ 的某组基下对应于对角阵 $\left[\begin{array}{cc}\lambda E_{\dim A}&\\&\Lambda_{\dim W_A}\end{array}\right]$
    \end{itemize}

    \item [4.] 63页 习题4
    
    \begin{itemize}
        \item [(1)] 这里利用 Jordan 标准型的唯一性来说明:假设对于 $V$ 的某个非平凡的 $\sigma$-子空间 $W$,存在 $\sigma$-子空间直和补 $W^\prime$,那么假设 $w_1, \cdots w_r$ 是 $W$ 的一组基,$0<r<n$, $w_{r+1}, \cdots, w_n$ 是 $W^\prime$ 的一组基,则有 $\sigma (w_1, \cdots, w_r, w_{r+1},\cdots, w_n) = (w_1, \cdots, w_r, w_{r+1},\cdots, w_n)\left[\begin{array}{cc}A&\\&B\end{array}\right]$. 其中 $A$ 是 $r$ 阶方阵, $B$ 是 $n-r$ 阶方阵。于是 $J = \lambda E_n + E_{12} + \cdots + E_{n-1,n}$ 相似于 $\left[\begin{array}{cc}A&\\&B\end{array}\right]$ 以及相似于 它的 Jordan 标准型 $\left[\begin{array}{cc}J_A&\\&J_B\end{array}\right]$. 这与 方阵的Jordan标准型 唯一相矛盾,所以对于 $V$ 的任意非平凡 $\sigma$-子空间 不存在 $\sigma$-子空间直和补。
        \item [(2)] 不妨将这组基显式地设出来 $\{\alpha_i\}$, $\sigma(\alpha_1, \cdots, \alpha_n) = (\alpha_1,\cdots,\alpha_n) J$. 下面归纳地证明:
        
        若 $W$ 是 $V$ 的一个维数 不少于 $k$ 的 $\sigma$-子空间,那么 $\alpha_i \in W, \forall i\leq k$.

        \begin{proof}
            ~\\
            \begin{itemize}
                \item [i.] 对于 $k=1$, 由于 $W$ 维数至少为 $1$, 故存在 $W \ni v = \sum_i c_i \alpha_i$, 使得至少有一个 $c_i\ne 0$. 设 $j$ 是使 $c_j\ne 0$ 成立的最大下标。考虑到 $W$ 也是 $\tau = \sigma - \lambda I_V$ 不变的,又有 $\tau \alpha_1 = 0, \tau \alpha_{i+1} = \alpha_i, i\geq 1$, 故 $\tau^{j-1} v = c_j\alpha_1 \in W$, 从而得到 $\alpha_1 \in W$.
                \item [ii.] 若 $W$ 是 $V$ 的一个维数不少于 $k+1$ 的 $\sigma$-子空间,那么存在 $W\ni v = \sum_i c_i\alpha_i$, 使得至少有 $k+1$ 个 $c_i\ne 0$ (否则$W$的维数不超过$k$). 设 $j$ 是使得 $c_j\ne 0$ 成立的最大下标, 显然有 $j\geq k+1$。同样地,构造 $\tau = \sigma - \lambda I_V$, 显然 $W$ 也是 $\tau$不变的。由于 $\tau^{j-k-1} v = c_j \alpha_{k+1} + \sum_{i > j - k - 1} c_i\alpha_{i - (j - k - 1)} = c_j\alpha_{k+1} + v^\prime\in W$, 其中 $v^\prime \in \mathtt{span}\{\alpha_1,\cdots, \alpha_k\}$. 由归纳假设知,$\mathtt{span}\{\alpha_1,\cdots, \alpha_k\} \subseteq W$ (由 $W$ 维数不少于 $k$ 推知), 故而 $\alpha_{k+1}\in W$.
            \end{itemize}
        \end{proof}

        由上述命题可知,$\sigma$-子空间有且仅有 $\{0\}$ 以及 $\mathtt{span}\{\alpha_1,\cdots,\alpha_i\}$ 其中 $1\leq i\leq n$.

        事实上由这一命题也能推出本题 (1) 问中的结论。
    \end{itemize}
    
    \item [5.] 63页 习题6
    
    由本次作业第2题(60页习题8)可知,$\sigma$ 是正规变换,当且仅当它在某一组标准正交基 $\alpha_1, \cdots, \alpha_n$ 下对应的矩阵 $A$ 是正规矩阵。而由正规矩阵基本定理可知,正规矩阵 $A$ 可以酉对角化,即存在酉矩阵 $U$, $UAU^\star = \Lambda$. 所以这等价于 $\sigma$ 在某组标准正交基,即 $(\alpha_1, \cdots, \alpha_n)U^\star$ 下, 对应于对角矩阵 $\Lambda = UAU^\star$.

    那么由教材61页引理3.4.2可知,设$\lambda_1, \cdots, \lambda_r$ 是 $\sigma$ 的互异的特征值,则 $V = \bigoplus_{i=1}^r V_i$, 其中 $V_i = \{v\in V| \sigma v = \lambda_i v\}$. 再仿照教材62页 命题3.4.3 的方法, 可将 $W$ 分解为 $W = \bigoplus_i (W\cap V_i)$. 考虑到 对于 $W\cap V_i$, 其在 $V_i$ 中存在正交补 $(W\cap V_i)^\perp$ (对$V_i$中的特征向量做正交化即可构造), 显然 $(W\cap V_i)^\perp$ 也是 $\sigma$ 不变的。于是 可以构造 $W^\perp = \bigoplus_i (W\cap V_i)^\perp$, 且它是 $\sigma$-子空间。

    \item [1.] 习题 1
    
    $A = \left[\begin{array}{cccc}3&-1&0&0\\1&1&0&0\\3&0&5&-3\\4&-1&3&-1\end{array}\right]$ 首先求其Jordan标准型:容易计算 $\det(xE-A) = (x-2)^4$. 考虑 
    
    $B = A - 2E = \left[\begin{array}{cccc}1&-1&0&0\\1&-1&0&0\\3&0&3&-3\\4&-1&3&-3\end{array}\right]$, $B^2 = O$. 求得 $Bx=0$ 的解空间上的一组基 $\alpha_1 = [-1, -1, 0, -1]^\top, \alpha_2=[0, 0, 3, 3]^\top$. 容易构造 $\beta_1 = [0, 1, 0, 0]^\top, \beta_2 = [0, 0, 1, 0]^\top$ 使 $B\beta_1 = \alpha_1, B\beta_2 = \alpha_2$. 容易验证 $\beta_1, \beta_2, \alpha_1, \alpha_2$ 线性无关,故构造得到 $S = [\alpha_1, \beta_1, \alpha_2, \beta_2] = \left[\begin{array}{cccc}-1&0&0&0\\-1&1&0&0\\0&0&3&1\\-1&0&3&0\end{array}\right]$. 求得 $S^{-1} = \left[\begin{array}{cccc}-1&0&0&0\\-1&1&0&0\\-1/3&0&0&1/3\\1&0&1&-1\end{array}\right]$. 从而 $S^{-1}AS = J = \left[\begin{array}{cccc}2&1&&\\&2&&\\&&2&1\\&&&2\end{array}\right] = 2E + B_1$. 其中 $B_1$ 是幂零的Jordan阵, $B_1^2=O$.

    \begin{itemize}
        \item [(1)] $\sin(2 + x) = \sin(2) + \cos(2) \cdot x - \dfrac{\sin(2)}{2}x^2 + \cdots$, $\sin(J) = \sin(2 E + B_1) = \sin(2)E + \cos(2) B_1$. 
        
        $\sin(A) = S \cdot \sin(J) \cdot S^{-1} = S \cdot \left[\begin{array}{cccc}\sin(2)&\cos(2)&&\\&\sin(2)&&\\&&\sin(2)&\cos(2)\\&&&\sin(2)\end{array}\right] \cdot S^{-1}$

        由于 $\sin A = (\sin 2) SES^{-1} + (\cos 2) SB_1S^{-1} = (\sin 2) E + (\cos 2) (A - 2E)$,

        具体计算可得 $\sin(A) = \left[\begin{array}{cccc}\sin 2 + \cos 2&-\cos 2&0&0\\\cos 2&-\cos 2 + \sin 2 & 0 & 0\\3\cos 2&0&3\cos 2 + \sin 2& -3\cos 2\\4\cos 2&-\cos 2&3\cos 2&\sin 2 - 3\cos 2\end{array}\right]$.

        
        \item [(2)] $\mathtt{e}^J = \mathtt{e}^{2}(E + B_1) = \mathtt{e}^{2} \left[\begin{array}{cccc}1&1&&\\&1&&\\&&1&1\\&&&1\end{array}\right]$, $\mathtt{e}^{A} = S\cdot \mathtt{e}^{J}\cdot S^{-1}$.

        由于 $\mathtt{e}^A = \mathtt{e}^2 \cdot S(E+B_1)S^{-1} = \mathtt{e}^2 (E + A - 2E) = \mathtt{e}^2 (A - E)$,

        具体计算可得 $\mathtt{e}^A = \mathtt{e}^2\cdot \left[\begin{array}{cccc}2&-1&0&0\\1&0&0&0\\3&0&4&-3\\4&-1&3&-2\end{array}\right]$.

    \end{itemize}

    \item [1.] 94页 习题1
    
    为避免符号混淆,将新定义的范数记为 $||-||_m$, $||\alpha||_m:=||A\alpha||$, $A$列满秩.
    \begin{itemize}
        \item [正定性:] 由于 $||-||$是向量范数,故而 $||\alpha||_m = ||A\alpha|| \geq 0$. $A$ 可作分解 $A = QR$, 其中 $Q\in\mathbb{F}^{m\times n}$ 是列酉阵 $Q^\star Q=E_n$, $R$ 是满秩上三角矩阵。故而 $||\alpha||_m=0\Rightarrow ||A\alpha||=0 \Rightarrow A\alpha=QR\alpha=0 \Rightarrow \alpha = R^{-1}Q^\star QR\alpha = 0$.
        \item [齐次性:] $||k\alpha||_m = ||A(k\alpha)|| = |k|\cdot||A\alpha|| = |k|\cdot||\alpha||_m$.
        \item [三角不等式:] $||\alpha+\beta||_m = ||A(\alpha+\beta)|| \leq ||A\alpha|| + ||A\beta|| = ||\alpha||_m + ||\beta||_m$.
    \end{itemize}

    \item [2.] 94页 习题2
    
    $||\alpha||_p =\left(\sum_{i}^{n} |\alpha_i|^p\right)^{1/p}$. 设 $m = \max_{i} |\alpha_i| = ||\alpha||_\infty$, 则 $1^{1/p}\leq\dfrac{||\alpha||_p}{||\alpha||_\infty} = \left(\sum_{i}^{n} \left(\dfrac{|\alpha_i|}{m}\right)^p\right)^{1/p}\leq n^{1/p}$.

    考虑到 $\lim_{p\to\infty} 1^{1/p} = 1, \lim_{p\to\infty} n^{1/p} = 1$, 所以 $\lim_{p\to\infty} \dfrac{||\alpha||_p}{||\alpha||_\infty} = 1$, 所以 $\lim_{p\to\infty} ||\alpha||_p = ||\alpha||_{\infty}$.

    \item [3.] 94页 习题3
    
    \begin{itemize}
        \item [(1)] 首先验证 $||\alpha||_A = \sqrt{\alpha^\star A\alpha}$ 是一个向量范数,其中 $A$ 是正定Hermite矩阵
        \begin{itemize}
            \item [正定性:] 由 $A$ 是Hermite矩阵可知 $||\alpha||_A\geq 0$ 且 $||\alpha||_A = 0 \rightarrow \alpha = 0$.
            \item [齐次性:] $||k\alpha||_A = \sqrt{k^2\alpha^\star A\alpha} = |k|\cdot\sqrt{\alpha^\star A\alpha} = |k|\cdot||\alpha||_A$.
            \item [三角不等式:] 对正定的Hermite矩阵 $A$ 作 Cholesky 分解,$A = LL^\star$. 由 Cauchy-Schwarz 不等式, $\forall x, y$,
            
            $[L^\star x, L^\star y] \leq \sqrt{[L^\star x, L^\star x]}\sqrt{[L^\star y, L^\star y]}$, 展开即得 $x^\star Ay = x^\star LL^\star y \leq \sqrt{x^\star LL^\star x y^\star LL^\star y}=\sqrt{x^\star Axy^\star Ay}$. 所以 $(||x||_A + ||y||_A)^2 - ||x+y||_A^2 = 2\left(\sqrt{x^\star Ax}\sqrt{y^\star Ay} - x^\star Ay\right) \geq 0$, 即 $||x||_A + ||y||_A \geq ||x + y||_A$.
        \end{itemize}
        \item [(2)] 为说明“当$A$遍历全部$n$阶Hermite正定矩阵时,$||-||_A$遍历全部由$V$上内积确定的范数”,只需要说明,对于任意一个 $V$ 上内积 $[-,-]$,存在对应的Hermite正定矩阵$A$使得 $\forall \beta\in V, [\beta, \beta] = ||\beta||_A^2$.
        
        考虑 $V=\mathbb{C}^{n\times 1}$的一组基 $e_1, e_2, \cdots, e_n$, 其中 $e_i$ 表示第 $i$ 个分量为 $1$ 其余分量全为 $0$ 的向量。对于任意的 $\beta \in V$, 有 $\beta = (e_1, e_2, \cdots, e_n)\beta$. 对于内积 $[-, -]$ 有 $[\beta, \beta] = [\sum_{i}\beta_ie_i, \sum_{i}\beta_ie_i]=\sum_{i}\sum_{j}\beta_i^\star [e_i, e_j] \beta_j = \beta^\star A\beta$, 其中 $A = \{[e_i, e_j]\}_{ij}$,即 $A$ 是由基$\{e_i\}$之间的内积取值组成的矩阵。由内积的正定性可知 $A$ 是正定矩阵,由内积的对称性可知 $A$ 是 Hermite 矩阵。
    \end{itemize}

    \item [4.] 101页 习题1 
    
    显然,由于 $||A||_{M_1}$ 相当于矩阵拉平后的向量的 $1-$范数,所以 $||-||_{M_1}$  满足向量范数的要求. 
    
    且有 $||AB||_{M_1} = \sum_{i,j} |(AB)_{ij}| = \sum_i\sum_j\sum_k |A_{ij}||B_{jk}| \leq \sum_{i}\sum_{j} |A_{ij}|||B||_{M_1} = ||A||_{M_1} ||B||_{M_1}$.

    \item [5.] 101页 习题2 
    
    $0 \leq ||A^k|| \leq ||A||^k$. 又知 $||A|| < 1$, 所以 $\lim_{k\to\infty} ||A||^k = 0$, 所以 $\lim_{k\to\infty} ||A^k|| = 0$, 故 $\lim_{k\to\infty} A^k = 0$.

    \item [6.] 101页 习题3
    \begin{itemize}
        \item [(1)]
        $||UA||_F = \sqrt{\mathtt{tr}\left((UA)^\star(UA)\right)}=\sqrt{\mathtt{tr}(A^\star U^\star UA)} = \sqrt{\mathtt{tr}(A^\star A)} = ||A||_F$; 
    
        $||AU||_F = \sqrt{\mathtt{tr}\left((AU)^\star (AU)\right)}=\sqrt{\mathtt{tr}(U^\star A^\star AU)} = \sqrt{\mathtt{tr}(UU^\star A^\star A)} = \sqrt{\mathtt{tr}(A^\star A)} = ||A||_F$.
        \item [(2)]
        正规矩阵 $N$ 可酉相似对角化,$N = U^\star\mathtt{diag}\{\lambda_1, \cdots, \lambda_n\}U$. 从而有 $||N||_F = ||U^\star\mathtt{diag}\{\lambda_1, \cdots, \lambda_n\}U||_F = ||\mathtt{diag}\{\lambda_1, \cdots, \lambda_n\}||_F = \sqrt{\sum_i \bar{\lambda}_i\lambda_i} = \sqrt{\sum_i |\lambda_i|^2}$.

    \end{itemize}
    

    \item [7.] 101页 习题4
    \begin{itemize}
        \item [(1)]
        对于任一矩阵 $A$, $||A||_2 = \sqrt{\rho(A^\star A)}$. 其中 $\rho(-)$ 表示谱半径。 所以 
    
        $||UA||_2=\sqrt{\rho(A^\star U^\star U A)} = \sqrt{\rho(A^\star A)} = ||A||_2$
    
        $||AU||_2 = \sqrt{\rho(U^\star A^\star A U)} = \sqrt{\rho(A^\star A)} = ||A||_2$, 因为 $\det(xE - U^\star A^\star AU) = \det(xE - A^\star A)$.
    
        所以 $||-||_2$ 是酉不变的。
        \item [(2)]
        
        $N = U^\star\mathtt{diag}\{\lambda_1,\cdots,\lambda_n\}U$, $||N||_2=||\mathtt{diag}\{\lambda_1,\cdots,\lambda_n\}||_2=\sqrt{\max_{i} |\lambda_i|^2} = \max_{i} |\lambda_i|$.
    \end{itemize}
    
    \item [1.] 习题6
    
    \begin{itemize}
        \item [(1)] $||A||_{M_1} := \sum_{1\leq i, j\leq n} |a_{ij}|$. 由例4.2.8(1)可知 $||A||_1 = \max_{1\leq j\leq n} \sum_{i=1}^{n} |a_{ij}|$, 
        
        从而有 $||A||_1 \leq \sum_{1\leq j\leq n}\sum_{1\leq i\leq n} |a_{ij}| = ||A||_{M_1}$.
        
        所以对于任意的$A\in M_{n}(\mathbb{F}), v\in\mathbb{F}^n$, $||Av||_1 \leq ||A||_1||v||_1 \leq ||A||_{M_1}||v||_1$, 即$||-||_{M_1}$ 与 $||-||_1$ 相容。

        \item [(2)] 由例4.2.8(2)知, $||A||_2 = \sqrt{\rho(A^\star A)} = \sqrt{\lambda_{\mathtt{max}}(A^\star A)}$. 而对 Frobenius 范数 $||A||_F := \sqrt{\sum_{i,j} \bar a_{ij}a_{ij}}=\sqrt{\mathtt{tr}(A^\star A)} = \sqrt{\sum_i \lambda_i(A^\star A)} \geq \sqrt{\lambda_{\mathtt{max}}(A^\star A)} = ||A||_2$.
        
        所以 $||Av||_2 \leq ||A||_2||v||_2 \leq ||A||_F||v||_2$, 即 $||-||_F$ 与 $||-||_2$ 相容。

        \item [(3)] $||A||_{M_\infty} := n\cdot \max_{i,j} |a_{ij}|$.
        \begin{itemize}
            \item [$p=1$:] 由例4.2.8(1)知, $||A||_1 = \max_{j} \sum_{i} |a_{ij}| = \sum_{i} |a_{ij^\prime}| \leq n\cdot \max_i |a_{ij^\prime}| \leq n\cdot \max_{i,j} |a_{ij}| = ||A||_{M_\infty}$, 同(1)(2)的步骤可知,$||-||_{M_\infty}$ 与 $||-||_1$ 相容
            \item [$p=2$:] 由(2)知, $||A||_2 \leq ||A||_F = \sqrt{\sum_{i,j} |a_{ij}|^2}\leq \sqrt{{n^2\cdot (\max_{i,j} |a_{ij}|)^2}} = ||A||_{M_\infty}$. 再同(1)(2)可得 $||-||_{M_\infty}$ 与 $||-||_2$ 相容
            \item [$p=\infty$:] 由例4.2.8(3)知, $||A||_\infty = \max_{i} \sum_{j} |a_{ij}|$, 完全仿照 $p=1$ 的情况即得 $||-||_{M_\infty}$ 与 $||-||_{\infty}$ 相容
        \end{itemize}
    \end{itemize}
    
    \item [2.] 习题7 
    
    考虑对 $A\in M_n(\mathbb{C})$ 作奇异值分解,即 $A = USV$, 其中 $U, V$ 是酉矩阵,$S = \mathtt{diag}(s_A)$, $s_A(1), \cdots, s_A(n)$ 是 $A$ 的全体奇异值。由于 $||-||$ 是酉不变的,所以 $||A|| = ||USV|| = ||S||$. 

    按如下方式定义 $\mathbb{R}^n$ 上的范数 $N$: $N(v) := ||\mathtt{diag}(v)||$. 由于矩阵范数 $||-||$ 满足向量范数的各个要求,所以这样定义出来的 $N$ 也满足向量范数的要求,且显然有 $||A|| = ||S|| = ||\mathtt{diag}(s_A)|| = N(s_A)$.
    
    \item [3.] 习题8 
    
    对于任意的 $x, y \in \mathbb{R}^{+}, \lambda \in [0, 1]$, $N\left(\lambda x + (1-\lambda)y\right) = ||A + \left( \lambda x + (1-\lambda) y \right) B|| = ||\lambda A + \lambda xB + (1-\lambda)A + (1-\lambda)y B|| \leq ||\lambda A + \lambda x B|| + ||(1-\lambda)A + (1-\lambda)yB|| = \lambda N(x) + (1-\lambda) N(y) $, 这就验证了 $N(x)$ 是凸函数
    

\end{enumerate}

\end{spacing}

\end{document}