\documentclass{article}
\usepackage{ctex}
\usepackage{amsmath,amscd,amsbsy,amssymb,latexsym,url,bm,amsthm}
\usepackage{epsfig,graphicx,subfigure}
\usepackage{enumitem,balance,mathtools}
\usepackage{wrapfig}
\usepackage{mathrsfs, euscript}
\usepackage[usenames]{xcolor}
\usepackage{hyperref}
\usepackage{caption}
\usepackage{setspace}
%\usepackage{subcaption}
\usepackage{float}
\usepackage{listings}
%\usepackage{enumerate}
%\usepackage{algorithm}
%\usepackage{algorithmic}
%\usepackage[vlined,ruled,commentsnumbered,linesnumbered]{algorithm2e}
\usepackage[ruled,lined,boxed,linesnumbered]{algorithm2e}
\usepackage{tikz}

\newtheorem{theorem}{Theorem}[section]
\newtheorem{lemma}[theorem]{Lemma}
\newtheorem{proposition}[theorem]{Proposition}
\newtheorem{corollary}[theorem]{Corollary}
\newtheorem{exercise}{Exercise}[section]
\newtheorem*{solution}{Solution}

\renewcommand{\thefootnote}{\fnsymbol{footnote}}
\renewenvironment{solution}[1][Solution]{~\\ \textbf{#1.}}{~\\}

\newcommand{\prob}{\mathtt{Pr}}

\newcommand{\postscript}[2]
{\setlength{\epsfxsize}{#2\hsize}
\centerline{\epsfbox{#1}}}

\renewcommand{\baselinestretch}{1.0}
\SetKwFor{Function}{function}{:}{end}
\setlength{\oddsidemargin}{-0.365in}
\setlength{\evensidemargin}{-0.365in}
\setlength{\topmargin}{-0.3in}
\setlength{\headheight}{0in}
\setlength{\headsep}{0in}
\setlength{\textheight}{10.1in}
\setlength{\textwidth}{7in}

\title{矩阵理论 作业6}
\author{刘彦铭 \ \ ID: 122033910081}
\date{Last Edited:\ \today}

\begin{document}

\maketitle

2.7节 习题2、3、12.

\begin{spacing}{1.5}

\begin{itemize}
    \item [习题2]
    
    $A = \left[\begin{array}{ccc}1&1&0\\0&1&5\end{array}\right]$, $AA^\top = \left[\begin{array}{cc}2&1\\1&26\end{array}\right]$,容易验证这是一个正定的 Hermite 矩阵,所以它的 Cholesky 分解存在. 证明了存在性后,Cholesky 分解的具体构造较为容易,这里直接给出结果:
    
    $L = \left[\begin{array}{cc}\sqrt{2} & 0 \\ \sqrt{1/2} & \sqrt{51/2}\end{array}\right]$, $AA^\top = LL^\top$.

    \item [习题3] 
    
    \begin{itemize}
        \item [(1)] 首先,容易验证 $AA^\star$ 是一个半正定的 Hermite 矩阵, 那么存在酉矩阵$U_m\in\mathbb{C}^{n\times n}$,使得
        
        $AA^\star = U_m \mathtt{diag}(\lambda_1, \dots, \lambda_r, 0, \dots, 0) U_m^\star$. 其中 $r = r(A) = r(A^\star) = r(AA^\star)$. 
        
        对 $A$ 进行奇异值分解,有 $A = U_m\left[\begin{array}{cc}\Lambda_r&0\\0&0\end{array}\right]_{m\times n} V_n$,其中 $V_n$ 是另一个 $n$ 阶酉矩阵, $\Lambda_r = \mathtt{diag}(\lambda_1^{1/2},\dots, \lambda_r^{1/2})$. 方便起见,记 $\Lambda = \mathtt{diag}(\lambda_1, \dots, \lambda_r, 0,\dots 0)$. 对SVD分解中的$m\times n$准对角矩阵进行分块可得: 
        
        $$A = U_m \left[ \Lambda^{1/2} ; O_{m\times (n - m)} \right] V_n = U_m \Lambda^{1/2}\left[E_m; O\right]V_n = U_m \Lambda^{1/2} U_m^\star U_m [E_m; O] V_n = PU_{m\times n} $$

        其中 $P = U_m \Lambda^{1/2} U_m^\star = (AA^\star)^{1/2}$, $U_{m\times n} = U_m [E_m; O] V_n = [V_m; O] V_n$, 可以验证 $U U^\star = [V_m; O] V_n V_n^\star [V_m; O]^\star = V_mV_m^\star = E_m$. 

        \item [(2)] $r(A) = m$ 时,$AA^\star$ 是满秩方阵, $P = (AA^\star)^{1/2}$ 也是满秩方阵,故而 $U = P^{-1}A$ 唯一确定。

    \end{itemize}

    \item [习题12] 证明关于方阵 $A\in\mathbb{M}_n$ 的下列三个命题的等价性:
    \begin{itemize}
        \item [(1)] 存在正整数 $k \geq 1$, 使得 $A^k = 0$ ;
        \item [(2)] 对于任意正整数 $m\geq 1$, $tr(A^m) = 0$ ;
        \item [(3)] 对于任意正整数 $m$, $1\leq m \leq n$, $tr(A^m) = 0$ .
    \end{itemize}
    \begin{proof}
        为方便讨论矩阵的迹, 根据 Schur 引理,对 $A$ 做分解: 存在酉矩阵 $U\in\mathbb{M}_n$, 和 上三角矩阵 $R\in\mathbb{M}_n$ 使得 $A = URU^\star$. 从而有 $A^k = UR^kU^\star, \forall k\in\mathbb{N}^*$. 通过简单的数学归纳可以证明 $R^k$ 是上三角矩阵,且 $(R^k)_{ii} = (R_{ii})^k$. 下面按照 (1) $\Rightarrow $ (2) $\Rightarrow $ (3) $\Rightarrow$ (1) 的顺序来证明等价性:
        \begin{itemize}
            \item [(1) $\Rightarrow$ (2)] $A^k = UR^kU^\star = 0$, 其中酉矩阵 $U$ 可逆, 所以 $R^k = 0$. 于是对于任意的 $1\leq i\leq n, i\in\mathbb{N}$, $(R^k)_{ii} = (R_{ii})^k = 0$, 所以 $R_{ii} = 0$ 即 $R$ 的对角线元素均为 $0$. 所以 对正整数 $m\geq 1$, $tr(R^m) = \sum_i R_{ii}^m = 0$, $tr(A^m) = tr(UR^mU^\star) = tr(R^mU^\star U) = tr(R^m) = 0$.
            \item [(2) $\Rightarrow (3)$] 显然
            \item [(3) $\Rightarrow (1)$] 对于任意的 $1\leq m\leq n$, $tr(A^m) = tr(R^m) = \sum_{i} R_{ii}^m = 0$. 假设 $R_{ii}, 1\leq i\leq n$ 不全为 $0$, 则可取出其中不为 $0$ 的项,去重后得到 $r_1, r_2, \cdots, r_t$, $1\leq t \leq n$. 从而有方程组
            $$\left[\begin{array}{cccc}
                r_1 & r_2 & \cdots & r_t\\
                r_1^2 & r_2^2 & \cdots & r_t^2\\
                \vdots & \vdots & \ddots & \vdots\\
                r_1^{t} & r_2^{t} & \cdots & r_t^{t} 
            \end{array}\right]
            \left[\begin{array}{c}
                n_1 \\ n_2 \\ \vdots \\ n_t
            \end{array}\right] = 
            \left[\begin{array}{cccc}
                1 & 1 & \cdots & 1\\
                r_1^1 & r_2^1 & \cdots & r_t^1\\
                \vdots & \vdots & \ddots & \vdots\\
                r_1^{t-1} & r_2^{t-1} & \cdots & r_t^{t-1} 
            \end{array}\right]
            \left[\begin{array}{cccc}
                r_1 & & & \\
                & r_2 & & \\
                & & \ddots & \\
                & & & r_t
            \end{array}\right]
            \left[\begin{array}{c}
                n_1 \\ n_2 \\ \vdots \\ n_t
            \end{array}\right] = 
            \left[\begin{array}{c}
                0 \\ 0 \\ \vdots \\ 0
            \end{array}\right]$$
            其中 $n_i$ 表示 $r_i$ 去重前的出现次数,应有 $r_i\ne r_j, \forall i\ne j$ 以及 $n_i > 0, r_i \ne 0, \forall i$.
            故而此时 方程中的 Vandermonde 矩阵和对角矩阵均可逆,从而$[n_1, n_2, \cdots, n_t]^\top$应为 $0$ 向量,矛盾。所以 $R_{ii}$ 不全为 $0$ 的假设不成立,从而得到 $R$ 是对角线全为 $0$ 的上三角矩阵。即 $R_{ij} = 0, \forall i < j + 1$. 
            
            下归纳证明 $(R^{k})_{ij} = 0, \forall i < j + k$ : (1) 对于 $k=1$ 成立;(2) $(R^k)_{ij} = 0, \forall i < j + k$ $\Rightarrow$ $(R^{k+1})_{ij} = (R^k R)_{ij} = \sum_t (R^k)_{it} R_{tj}$ . 当 $i < j + k + 1$ 时, $i \geq t + k$ 与 $t \geq j + 1$ 不能同时成立,故 $(R^k)_{it}$ 与 $R_{tj}$ 中至少有一个为 $0$, 从而推出 $(R^{k+1})_{ij} = 0, \forall i < j + (k + 1)$.

            对于任意 $1\leq i,j\leq n$, 有 $i < j + n$, 于是 $(R^n)_{ij} = 0$, 所以 $R^n = 0$, $A^n = UR^n U^\star = 0$.

            注:这还说明了,如果存在正整数 $k$ 使得 $A^k = 0$, 那么存在 $k \leq n$ 使得 $A^k = 0$.
        \end{itemize}
    \end{proof}

\end{itemize}
    
\end{spacing}

\end{document}