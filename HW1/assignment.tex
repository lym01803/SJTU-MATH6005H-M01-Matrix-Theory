\documentclass{article}
\usepackage{ctex}
\usepackage{amsmath,amscd,amsbsy,amssymb,latexsym,url,bm,amsthm}
\usepackage{epsfig,graphicx,subfigure}
\usepackage{enumitem,balance,mathtools}
\usepackage{wrapfig}
\usepackage{mathrsfs, euscript}
\usepackage[usenames]{xcolor}
\usepackage{hyperref}
\usepackage{caption}
\usepackage{setspace}
%\usepackage{subcaption}
\usepackage{float}
\usepackage{listings}
%\usepackage{enumerate}
%\usepackage{algorithm}
%\usepackage{algorithmic}
%\usepackage[vlined,ruled,commentsnumbered,linesnumbered]{algorithm2e}
\usepackage[ruled,lined,boxed,linesnumbered]{algorithm2e}
\usepackage{tikz}

\newtheorem{theorem}{Theorem}[section]
\newtheorem{lemma}[theorem]{Lemma}
\newtheorem{proposition}[theorem]{Proposition}
\newtheorem{corollary}[theorem]{Corollary}
\newtheorem{exercise}{Exercise}[section]
\newtheorem*{solution}{Solution}

\renewcommand{\thefootnote}{\fnsymbol{footnote}}
\renewenvironment{solution}[1][Solution]{~\\ \textbf{#1.}}{~\\}

\newcommand{\prob}{\mathtt{Pr}}

\newcommand{\postscript}[2]
    {\setlength{\epsfxsize}{#2\hsize}
    \centerline{\epsfbox{#1}}}

\renewcommand{\baselinestretch}{1.0}
\SetKwFor{Function}{function}{:}{end}
\setlength{\oddsidemargin}{-0.365in}
\setlength{\evensidemargin}{-0.365in}
\setlength{\topmargin}{-0.3in}
\setlength{\headheight}{0in}
\setlength{\headsep}{0in}
\setlength{\textheight}{10.1in}
\setlength{\textwidth}{7in}

\title{矩阵理论 作业1}
\author{刘彦铭\ \ 学号:122033910081}
\date{编辑日期:\ \today}

\begin{document}

\maketitle

\begin{enumerate}
    \item Page 5 习题1
    
    从镜面反射变换的几何意义来看:
        \begin{figure}[htb]
            \centering 
            \begin{tikzpicture}
                \node (C) at (4, 4) {$C$};
                \node (A) at (8, 2) {$A$};
                \node (B) at (0, 2) {$B$};
                \node (O) at (4, 0) {$O$};
                \node (M) at (4, 2) {$M$};

                \draw [->] (A) -- node[right]{$(1, 1, 1, 1)^\top$} (O);
                \draw [->] (O) -- node[left]{$(1, 0, 0, \sqrt3)^\top$} (B);
                \draw [->] (O) -- (C);
            \end{tikzpicture}
        \end{figure}
    
    镜面方向上的 $\alpha$ 与 $\overrightarrow{OC} = \overrightarrow{OB} + \overrightarrow{OA} = \overrightarrow{OB} - \overrightarrow{AO} = (0, -1, -1, \sqrt3-1)^\top$ 同方向
    
    故单位向量 $\alpha=\dfrac{\overrightarrow{OC}}{|\overrightarrow{OC}|}=\dfrac{1}{\sqrt{6-2\sqrt3}}(0, -1, -1, \sqrt3 - 1)^\top$

    所以 
    $$B=E-2\alpha\alpha^T=\dfrac{1}{3 - \sqrt 3}\left[\begin{array}{cccc}
        3 - \sqrt 3 & 0 & 0 & 0 \\
        0 & 2 - \sqrt 3 & -1 & \sqrt 3 - 1 \\
        0 & -1 & 2 -\sqrt 3 & \sqrt 3 - 1 \\
        0 & \sqrt 3 - 1 & \sqrt 3 - 1 & \sqrt 3 - 1    
    \end{array}\right]$$

    \item Page 11 习题1 
    
    可将该矩阵分解为行变换矩阵和分块对角阵的乘积:
    \begin{equation*}
        \left[
            \begin{array}{ccc}
                0 & B & 0 \\
                E & 0 & 0 \\
                0 & 0 & A \\
            \end{array}
        \right]
        = 
        \left[
            \begin{array}{ccc}
                0 & E & 0 \\
                E & 0 & 0 \\
                0 & 0 & E \\
            \end{array}
        \right]
        \times
        \left[
            \begin{array}{ccc}
                E & 0 & 0 \\
                0 & B & 0 \\
                0 & 0 & A \\
            \end{array}
        \right]
    \end{equation*}
    故其逆矩阵为 
    \begin{equation*}
        \left[
            \begin{array}{ccc}
                0 & B & 0 \\
                E & 0 & 0 \\
                0 & 0 & A \\
            \end{array}
        \right]^{-1}
        =
        \left[
            \begin{array}{ccc}
                E & 0 & 0 \\
                0 & B^{-1} & 0 \\
                0 & 0 & A^{-1} \\
            \end{array}
        \right]
        \times
        \left[
            \begin{array}{ccc}
                0 & E & 0 \\
                E & 0 & 0 \\
                0 & 0 & E \\
            \end{array}
        \right]
        =
        \left[
            \begin{array}{ccc}
                0 & E & 0 \\
                B^{-1} & 0 & 0 \\
                0 & 0 & A^{-1} \\
            \end{array}
        \right]
    \end{equation*}
    行列式:
    \begin{equation*}
        \left|
            \begin{array}{ccc}
                0 & B & 0 \\
                E & 0 & 0 \\
                0 & 0 & A \\
            \end{array}
        \right|
        = 
        \left|
            \begin{array}{ccc}
                0 & E & 0 \\
                E & 0 & 0 \\
                0 & 0 & E \\
            \end{array}
        \right|
        \times
        \left|
            \begin{array}{ccc}
                E & 0 & 0 \\
                0 & B & 0 \\
                0 & 0 & A \\
            \end{array}
        \right|
        = (-1)^{n}\cdot \det(A)\cdot\det(B)
    \end{equation*}
    \item Page 11 习题3
    
    注意到对该方阵进行行变换可以得到 
    \begin{equation*}
        \left[
            \begin{array}{cc}
                E & 0 \\
                -C & E \\
            \end{array}
        \right]
        \times
        \left[
            \begin{array}{cc}
                A^{-1} & 0 \\
                0 & E_m \\
            \end{array}
        \right]
        \times
        \left[
            \begin{array}{cc}
                A & B \\
                C & D_m
            \end{array}
        \right]
        = 
        \left[
            \begin{array}{cc}
                E & A^{-1}B \\
                0 & D_m - CA^{-1}B \\
            \end{array}
        \right]
    \end{equation*}
    上式中左边两个行变换矩阵均满秩,故 $H=\left[\begin{array}{cc}A&B\\C&D_m\\\end{array}\right]$ 可逆的充要条件为
    \begin{equation*}
        \left|
            \begin{array}{cc}
                E & A^{-1}B \\
                0 & D_m - CA^{-1}B \\
            \end{array}
        \right|
        = \det(D_m - CA^{-1}B)\ne 0
    \end{equation*}
    (1) 由上述讨论知,充要条件是 $m$ 阶方阵 $D_m-CA^{-1}B$ 可逆;
    
    (2) 简单起见,令 $D^\prime=D_m-CA^{-1}B$, $H$可逆时, $D^{\prime-1}$ 唯一存在。
    注意到,运用行变换有:

    \begin{equation*}
        \left[
            \begin{array}{cc}
                E & -A^{-1}B \\
                0 & E \\
            \end{array}
        \right]
        \times
        \left[
            \begin{array}{cc}
                E & 0 \\
                0 & D^{\prime-1} \\
            \end{array}
        \right] 
        \times 
        \left[
            \begin{array}{cc}
                E & A^{-1}B \\
                0 & D^\prime \\
            \end{array}
        \right]
        = E_{n+m}
    \end{equation*}
    所以 
    \begin{spacing}{1.5}
    \begin{equation*}
        \begin{array}{ll}
            H^{-1} & = 
            \left[
                \begin{array}{cc}
                    E & -A^{-1}B \\
                    0 & E \\
                \end{array}
            \right]
            \times
            \left[
                \begin{array}{cc}
                    E & 0 \\
                    0 & D^{\prime-1} \\
                \end{array}
            \right] 
            \times
            \left[
                \begin{array}{cc}
                    E & 0 \\
                    -C & E \\
                \end{array}
            \right]
            \times
            \left[
                \begin{array}{cc}
                    A^{-1} & 0 \\
                    0 & E_m \\
                \end{array}
            \right] \\
            & = 
            \left[
                \begin{array}{cc}
                    A^{-1} + A^{-1}B(D_m-CA^{-1}B)^{-1}CA^{-1} & -A^{-1}B(D_m-CA^{-1}B)^{-1} \\
                    -(D_m-CA^{-1}B)^{-1}CA^{-1} & (D_m-CA^{-1}B)^{-1}\\
                \end{array}
            \right]
        \end{array}
    \end{equation*}
    \end{spacing}
    非常丑,但验证了一下应该是对的。左上角似乎和Woodbury公式的形式是一致的。
    
    \item Page 12 习题6
    
    考虑当 $A\in \mathbb{R}^{n\times n}$ 时,对任意列向量 $x^\prime=\left[x^\top; y\right]^\top\in\mathbb{R}^{(n+1)\times 1}$, 其中 $x\in\mathbb{R}^{n\times 1}, y\in\mathbb{R}$:

    计算得到

    \begin{spacing}{1.5}
        
    \begin{equation*}
        \begin{array}{ll}
            f(x^\prime)=x^{\prime\top}\left[\begin{array}{cc}A&k\alpha\\k\alpha^\top&1\\\end{array}\right]x^\prime 
            &= 
            \left[x^\top; y\right]\left[\begin{array}{cc}A&k\alpha\\k\alpha^\top&1\\\end{array}\right]\left[\begin{array}{c}x\\y\end{array}\right]
            \\
            &= x^\top Ax + 2ky(x^\top\alpha)+y^2
            \\
            &= \left(y + k(x^\top\alpha)\right)^2 + x^\top Ax - k^2(x^\top\alpha x^\top\alpha)\\
            &= \left(y + k(x^\top\alpha)\right)^2 + x^\top (A - k^2\alpha\alpha^\top) x

        \end{array}
    \end{equation*}

    \end{spacing}

    \begin{itemize}
        \item [-] 若 $A-k^2\alpha\alpha^\top$ 是正定矩阵,那么 $\left[\begin{array}{cc}A&k\alpha\\k\alpha^\top&1\\\end{array}\right]$ 正定。因为:
        \begin{itemize}
            \item [--] 一方面 $f(x^\prime)\geq 0$ 恒成立;
            \item [--] 另一方面, $f(x^\prime)=0$ 可以推出 $x=0, y=0, x^\prime=0$.
        \end{itemize}
        \item [-] 若 $A-k^2\alpha\alpha^\top$ 是半正定矩阵,那么 $\left[\begin{array}{cc}A&k\alpha\\k\alpha^\top&1\\\end{array}\right]$ 半正定。因为:
        \begin{itemize}
            \item [--] 一方面 $f(x^\prime)\geq 0$ 恒成立;
            \item [--] 另一方面, 由于存在$x\ne 0$ 使得 $x^\top (A-k^2\alpha\alpha^\top)x =0$, 取 $y=-k(x^\top\alpha)$ 即得到非零的 $x^\prime$ 使得 $f(x^\prime)=0$.
        \end{itemize}
        \item [-] 若 $A-k^2\alpha\alpha^\top$ 是不定矩阵,那么 $\left[\begin{array}{cc}A&k\alpha\\k\alpha^\top&1\\\end{array}\right]$ 也是不定的。因为:
        此时 $A-k^2\alpha\alpha^\top$ 存在小于 $0$ 的特征值,取 $x$ 为其对应的特征向量,取 $y=-k(x^\top\alpha)$ 即构造得到 $x^\prime$, $f(x^\prime) < 0$.
    \end{itemize}


    \item Page 13 习题7
    
    方便起见,令 $C=AB\in\mathbb{F}^{m\times r}$, 直接计算验证:对于任一 $1\leq k\leq r$,
    \begin{equation*}
        \begin{array}{ll}
            \sum_{1\leq i\leq m} C_{ik} & = \sum_{1\leq i\leq m} \sum_{1\leq j\leq n} A_{ij}\times B_{jk} \\
            & = \sum_{1\leq j\leq n} \sum_{1\leq i\leq m} A_{ij}\times B_{jk} \\
            & = \sum_{1\leq j\leq n} B_{jk}\times\left(\sum_{1\leq i\leq m} A_{ij}\right) \\
            & = \sum_{1\leq j\leq n} B_{jk} \times a \\ 
            & = ab
        \end{array}
    \end{equation*}

\end{enumerate}

\end{document}
