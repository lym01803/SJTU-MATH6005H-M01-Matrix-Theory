\documentclass{article}
\usepackage{ctex}
\usepackage{amsmath,amscd,amsbsy,amssymb,latexsym,url,bm,amsthm}
\usepackage{epsfig,graphicx,subfigure}
\usepackage{enumitem,balance,mathtools}
\usepackage{wrapfig}
\usepackage{mathrsfs, euscript}
\usepackage[usenames]{xcolor}
\usepackage{hyperref}
\usepackage{caption}
\usepackage{setspace}
%\usepackage{subcaption}
\usepackage{float}
\usepackage{listings}
%\usepackage{enumerate}
%\usepackage{algorithm}
%\usepackage{algorithmic}
%\usepackage[vlined,ruled,commentsnumbered,linesnumbered]{algorithm2e}
\usepackage[ruled,lined,boxed,linesnumbered]{algorithm2e}
\usepackage{tikz}

\newtheorem{theorem}{Theorem}[section]
\newtheorem{lemma}[theorem]{Lemma}
\newtheorem{proposition}[theorem]{Proposition}
\newtheorem{corollary}[theorem]{Corollary}
\newtheorem{exercise}{Exercise}[section]
\newtheorem*{solution}{Solution}

\renewcommand{\thefootnote}{\fnsymbol{footnote}}
\renewenvironment{solution}[1][Solution]{~\\ \textbf{#1.}}{~\\}

\newcommand{\prob}{\mathtt{Pr}}

\newcommand{\postscript}[2]
{\setlength{\epsfxsize}{#2\hsize}
\centerline{\epsfbox{#1}}}

\renewcommand{\baselinestretch}{1.0}
\SetKwFor{Function}{function}{:}{end}
\setlength{\oddsidemargin}{-0.365in}
\setlength{\evensidemargin}{-0.365in}
\setlength{\topmargin}{-0.3in}
\setlength{\headheight}{0in}
\setlength{\headsep}{0in}
\setlength{\textheight}{10.1in}
\setlength{\textwidth}{7in}

\title{矩阵理论\ 作业7}
\author{刘彦铭\ \ ID: 122033910081}
\date{Last Edited:\ \today}

\begin{document}

\maketitle

55页 1、3、4、5

\begin{spacing}{1.5}
    
\begin{itemize}
    \item [1.] 习题 1
    
        $\sigma: A \to AB^\top + BA$, 注意到 $BA = (A^\top B^\top)^\top = (AB^\top)^\top$, 所以  $AB^\top + BA$ 是$2$阶的实对称矩阵,仍在 $V$ 中。 

        \begin{itemize}
            \item [(1)] 对任意的 $\lambda_1, \lambda_2 \in\mathbb{R}$, $A_1, A_2\in V$, 有 $\sigma(\lambda_1 A_1 + \lambda_2 A_2) = (\lambda_1 A_1 + \lambda_2 A_2) B^\top + B(\lambda_1 A_1 + \lambda_2 A_2)= \lambda_1\cdot(A_1B^\top + BA_1) + \lambda_2\cdot (A_2B^\top + BA_2) = \lambda_1 \sigma(A_1) + \lambda_2 \sigma(A_2)$.
            \item [(2)] 由于 $\sigma(E_{11}) = 2E_{11}$, $\sigma(E_{12} + E_{21}) = -2 E_{11} + (E_{12} + E_{21})$, $\sigma(E_{22}) = - (E_{12} + E_{21})$. 故 $\sigma$ 在这组基下,对应于矩阵 $\left[\begin{array}{ccc}2&-2&0\\0&1&-1\\0&0&0\end{array}\right]$
            \item [(3)] 对于任意 $A\in V$, $A = [E_{11}, E_{12} + E_{21}, E_{22}]\cdot [c_1, c_2, c_3]^\top, c_i\in\mathbb{R}$,
            
            $\sigma A = \sigma[E_{11}, E_{12} + E_{21}, E_{22}] \cdot [c_1, c_2, c_3]^\top = [E_{11}, E_{12} + E_{21}, E_{22}] \cdot \left[\begin{array}{ccc}2&-2&0\\0&1&-1\\0&0&0\end{array}\right]\cdot [c_1, c_2, c_3]^\top = [E_{11}, E_{12} + E_{21}, E_{22}]\cdot \left[2c_1 - 2c_2, c_2 - c_3, 0\right]^\top$, 由此可知 $\{E_{11}, E_{12} + E_{21}\}$ 是像子空间 $\mathtt{im}(\sigma)$ 的一组基。
            
            \item [(4)] 延用 (3) 中的记号,由于 $E_{11}, E_{12} + E_{21}, E_{22}$ 线性无关,所以 $\sigma A = 0$ 当且仅当 $\left[\begin{array}{ccc}2&-2&0\\0&1&-1\\0&0&0\end{array}\right] \left[\begin{array}{c}c_1\\c_2\\c_3\end{array}\right] = 0$, 解为 $[c_1, c_2, c_3]^\top = [c_3, c_3, c_3]^\top, c_3\in\mathbb{R}$. 故核子空间 $\mathtt{ker}(\sigma)$ 的一组基是 $\{E_{11} + E_{12} + E_{21} + E_{22}\}$.
            \item [(5)] 是直和,且 $V = \mathtt{im}(\sigma) \oplus \mathtt{ker}(\sigma)$. 因为 $\mathtt{im} (\sigma)$ 与 $\mathtt{ker}(\sigma)$ 的基不交,且它们基的并是 $V$ 的一组基。
        \end{itemize}

    \item [2.] 习题 3
    
    设 $\{\eta_1, \eta_2, \cdots, \eta_n\}$ 是 $n$ 维空间 $\mathbb{F}^{n}$ 的一组基。对于线性变换 $\sigma$, 以及任意的 $\alpha\in \mathbb{F}^n$, $\alpha = [\eta_1, \eta_2, \cdots, \eta_n]\cdot c^\top$, 其中 $c = [c_1, c_2, \cdots, c_n] \in\mathbb{F}^n$. $\sigma(\alpha) = \sigma [\eta_1, \eta_2, \cdots, \eta_n]\cdot c^\top$. 由于 $\{\eta_i\}$ 是基,故存在矩阵 $A$ 使得 $\sigma [\eta_1, \cdots, \eta_n] = [\eta_1, \cdots, \eta_n] \cdot A$. 再记 $N = [\eta_1, \eta_2, \cdots, \eta_n]$, 显然 $N$ 可逆, 构造 $B = NAN^{-1}\in M_n(\mathbb{F})$, 有 $B\alpha = B(Nc^\top) = NAN^{-1}Nc^\top = NAc^\top = \sigma(\alpha)$.

    \item [3.] 习题 4
    
    \begin{itemize}
        \item [(1)] $V = W_1 \oplus W_2$, $\sigma: \alpha_1 + \alpha_2 \to \alpha_1$, 其中 $\alpha_i\in W_i$. 取 $W_1$ 的一组基 $(\alpha_1, \cdots, \alpha_r)$; 取 $W_2$ 的一组基 $(\beta_1, \cdots, \beta_s)$. 那么 $(\alpha_1, \cdots, \alpha_r, \beta_1, \cdots, \beta_s)$ 是 $V$ 的一组基。对于 $V$ 的任意基 $(\gamma,\cdots, \gamma_r, \gamma_{r+1},\cdots, \gamma_{r+s})$, 有 $(\gamma,\cdots, \gamma_r, \gamma_{r+1},\cdots, \gamma_{r+s}) = (\alpha_1, \cdots, \alpha_r, \beta_1, \cdots, \beta_s) C$, 其中 $C\in M_{r+s}(\mathbb{F})$. 将 $\sigma$ 作用于其上:$$\begin{array}{ll}\sigma (\gamma,\cdots, \gamma_r, \gamma_{r+1},\cdots, \gamma_{r+s}) &= \sigma (\alpha_1, \cdots, \alpha_r, \beta_1, \cdots, \beta_s)\cdot C \\ &= (\alpha_1, \cdots, \alpha_r, \beta_1, \cdots, \beta_s) \left[\begin{array}{cc}E_r&0\\0&0\end{array}\right]C\\&=(\gamma,\cdots, \gamma_r, \gamma_{r+1},\cdots, \gamma_{r+s})\cdot C^{-1} \left[\begin{array}{cc}E_r&0\\0&0\end{array}\right]C\end{array}$$
        从而得到 $A = C^{-1} \left[\begin{array}{cc}E_r&0\\0&0\end{array}\right] C$, 显然有 $A^2 = A$.

        \item [(2)] 
        \begin{itemize}
            \item [(a)] 即 $\sigma_A: (\alpha_1, \cdots, \alpha_n) \beta \to (\alpha_1,\cdots,\alpha_n) (A\beta)$. 容易验证 对任意 $\lambda_1, \lambda_2 \in \mathbb{F}$, 以及 任意 $v_1, v_2\in V$, 有 $\exists \beta_1, v_1 = (\alpha_1, \cdots, \alpha_n) \beta_1$, $\exists \beta_2, v_2 = (\alpha_1, \cdots, \alpha_n) \beta_2$, $\sigma(\lambda_1 v_1+ \lambda_2 v_2) = (\alpha_1, \cdots, \alpha_n) (A\lambda_1\beta_1 + A\lambda_2\beta_2) = \lambda_1(\alpha_1, \cdots, \alpha_n)(A\beta_1) + \lambda_2(\alpha_1, \cdots, \alpha_n)(A\beta_2) = \lambda_1 \sigma(v_1) + \lambda_2 \sigma(v_2)$, 所以 $\sigma$ 是线性变换
            \item [(b)] $A^2 = A \Rightarrow A(A-E) = (A-E)A = 0$. 下考察 $r(A-E)$ 与 $r(A)$ 的关系. 设 $r(A) = r$, 则存在可逆矩阵 $P, Q\in M_n(\mathbb{F})$ 使得 $A = P\left[\begin{array}{cc}E_r&0\\0&0\end{array}\right]Q$, 由于 $A^2 = P\left[\begin{array}{cc}E_r&0\\0&0\end{array}\right]QP\left[\begin{array}{cc}E_r&0\\0&0\end{array}\right]Q=A$, 所以$QP$具有 $QP=\left[\begin{array}{cc}E_r&C_{12}\\C_{21}&C_{22}\end{array}\right]$的形式。$A - E = P\left[\begin{array}{cc}E_r&0\\0&0\end{array}\right]Q - PP^{-1}Q^{-1}Q = P\left(\left[\begin{array}{cc}E_r&0\\0&0\end{array}\right] - \left[\begin{array}{cc}E_{r}&C_{12}\\C_{21}&C_{22}\end{array}\right]^{-1}\right)Q$. 注意到 $\left[\begin{array}{cc}E_r&C_{12}\\C_{21}&C_{22}\end{array}\right]^{-1} = \left[\begin{array}{cc}E_r + C_{12}D^{-1}C_{21}&-C_{12}D^{-1}\\-D^{-1}C_{21}&D^{-1}\end{array}\right]$ , 其中 $D = C_{22} - C_{21}C_{12}$ 是($n-r$)阶可逆矩阵 (可逆性由 $QP$ 可逆推出). 所以: 
            $\left[\begin{array}{cc}E_r&0\\0&0\end{array}\right] - \left[\begin{array}{cc}E_r&C_{12}\\C_{21}&C_{22}\end{array}\right]^{-1}=\left[\begin{array}{c}-C_{12}\\E_{n-r}\end{array}\right] D^{-1} \left[C_{21}; -E_{n-r}\right]$, 从而 $r(A-E) = n-r$.

            下面验证 $A$ 的列向量空间和 $A-E$ 的解空间相同,即 $\mathtt{col}(A) = (A-E)^\perp$. $v\in \mathtt{col}(A)\Rightarrow \exists \gamma, v = A\gamma \Rightarrow (A-E)v = (A-E)A\gamma = 0 \Rightarrow v\in (A-E)^\perp$, 故 $\mathtt{col}(A)\subset (A-E)^\perp$. 又 $\dim(\mathtt{col}(A)) = r(A) = r = n - r(A-E) = \dim\left((A-E)^\perp\right)$, 所以 $\mathtt{col}(A) = (A-E)^\perp$. 同理可得,$\mathtt{col}(A-E) = A^\perp$.

            令 $W_1 = \{(\alpha_1,\cdots,\alpha_n)\beta | \forall \beta\in A^{\perp}\}$, $W_2 = \{(\alpha_1, \cdots, \alpha_n)\beta | \forall \beta\in (A-E)^\perp\}$ (或等价地,$W_1 = \{(\alpha_1,\cdots,\alpha_n)\beta | \forall \beta\in \mathtt{col}(A-E)\}$, $W_2 = \{(\alpha_1, \cdots, \alpha_n)\beta | \forall \beta\in \mathtt{col}(A)\}$), 则 $V = W_1 \oplus W_2$ 是满足题意的一个直和分解,验证如下:
            \begin{itemize}
                \item [(i)] 设 $v = (\alpha_1,\cdots,\alpha_n)\beta\in W_1\cap W_2$ 则 $\beta \in A^\perp \cap (A-E)^\perp$, 于是 $\beta = A\beta = 0$, $x = 0$. 所以 $W_1 \cap W_2 = 0$.
                \item [(ii)] $\dim(W_1) + \dim(W_2) = \dim(A^\perp) + \dim((A-E)^\perp) = (n - r) + r = n = \dim (V)$, 结合 (i) 知,$V = W_1 \oplus W_2$.
                \item [(iii)] $\forall \alpha = (\alpha_1,\cdots,\alpha_n)\beta\in W_1$, 有 $\sigma \alpha = (\alpha_1,\cdots,\alpha_n) (A\beta) = 0$ 因为 $\beta\in A^\perp, A\beta=0$. 类似地,$\forall \alpha = (\alpha_1,\cdots,\alpha_n)\beta \in W_2$, 有 $\sigma \alpha = (\alpha_1, \cdots, \alpha_n) (A\beta) = \alpha$ 因为 $\beta\in (A-E)^\perp, A\beta = \beta$. 
            \end{itemize}
            (好像$W_1$, $W_2$和题目中的顺序反了,算了不改了)
        \end{itemize}
    \end{itemize}

    \item [4.] 习题 5
    
    设线性变换 在 基 $\{\alpha_1, \alpha_2, \cdots, \alpha_n\}$ 下的对应的矩阵 为 $A$, 在基 $\{\beta_1, \beta_2, \cdots, \beta_n\}$ 下对应的矩阵为 $B$. 根据定义有:$\sigma (\alpha_1, \cdots, \alpha_n) = (\alpha_1, \cdots, \alpha_n) A$, $\sigma (\beta_1, \cdots, \beta_n) =(\beta_1, \cdots, \beta_n) B$. 考虑到同一线性空间的基之间能互相表示,故存在可逆矩阵 $Q$ 使得 $(\beta_1, \cdots, \beta_n) = (\alpha_1, \cdots, \alpha_n)Q$, 从而有 $$\begin{array}{ll}\sigma(\beta_1, \cdots, \beta_n) &= (\beta_1, \cdots, \beta_n)B = (\alpha_1,\cdots,\alpha_n)QB\\ &= \sigma [(\alpha_1, \cdots, \alpha_n) Q] = (\sigma\alpha_1, \cdots, \sigma\alpha_n)Q = (\alpha_1,\cdots,\alpha_n) AQ\end{array}$$
    故而 $(\alpha_1, \cdots, \alpha_n) (QB - AQ) = 0$, 由于基线性无关,故 $QB - AQ = 0 \Rightarrow A = QBQ^{-1}$, 即 $A$ 与 $B$ 相似。
    
\end{itemize}

\end{spacing}

\end{document}